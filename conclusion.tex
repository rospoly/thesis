%%% -*-LaTeX-*-

\chapter{Conclusions}
\label{sec:conclusion}
%
Floating-point arithmetic is ubiquitous in computers, and almost every programming language supports several floating-point data-types.
%
In many applications, floating-point arithmetic is perfectly adequate, we use it in every-day programs as a proxy for real arithmetic.
%
On the other hand, we cannot simply ignore roundoff errors stemming from floating-point computations, regardless of how rare or small they might be.
%
Hence, not only we have to be aware of these errors, but also, in applications like for example self-driving vehicles, there is the need to use rigorous --- that is to say sound --- techniques to bound roundoff errors, where we have an analytical model for the computation, thus we can provide guarantees about the error bounds.
%

In this thesis, we described several contributions to the rigorous roundoff error analysis of floating-point programs.
%
Moreover, we tested our tools on several case-studies from real-world applications.
%

First, we introduced FPRoCK a prototype of an SMT solver able to precisely reason about control-flow instabilities stemming from the use of floating-point arithmetic in computers.
%
We have a control-flow instability when the floating-point execution, diverges from the ideal execution in real arithmetic.
%
As opposed to the state-of-the-art, where the size of the instability region, from where an instability can be triggered, is over-approximated using conservative static analysis techniques, in FPRoCK we can mix floating-point and real arithmetics formulas together, and discharge the resulting query into the state-of-the-art for automated theorem provers. 
%
In this way, we can precisely reason about the exact input values leading to the instability. 
%
Moreover, in case the instability exists, we can get a model triggering the instability.
%
Due to its modularity, FPRoCK can be used as a back-end solver in any existing static analyzers to verify instabilities in conditionals, or alternatively, as a stand-alone tool to mix real and floating-point formulas in the same query.
%

Second, we introduce our framework for the design of robust micro-controllers, where we show how to embed the state-of-the-art for worst-case roundoff error analysis in the design process of micro-controllers.
%
Indeed, at design time, we can include the roundoff error stemming from the finite-precision implementation of the controller in the set of all the disturbances affecting the system (e.g. friction, wind).
%
Using our framework, the designer can precisely quantify the trade-off between the tolerance of the controller towards external disturbance which, among the others, includes roundoff errors, and the memory footprint of the micro-controller, in a context where typically the available memory is very limited (e.g. 32KB or 64KB).
%
Clearly, an high-precision implementation of the controller minimizes the roundoff error, while dramatically increase the memory requirements. 
%

Finally, we introduced our framework for computing rigorous probabilistic roundoff errors where, as opposed to the state-of-the-art, the roundoff error bounds hold for the \emph{average} computation rather than the \emph{worst-case} computation.
%
The intuition behind our approach is roundoff errors do not always achieve the worst-case magnitude. 
%

The main advantage in using average roundoff errors, as opposed to the worst-case counterparts, is we can trade some accuracy in the computations in favor of resource savings, like energy consumption and execution time.
%
This is precious not only on devices where the usage of the battery, and the dimension of the hardware, are primary concerns, like smart-phones and smart-watches, but also on all the applications where the worst-case approach is prohibitive from an implementation prospective, because it requires the use of very high-bit-width formats. 
%
In our framework we can precisely quantify which precision format is best suited for the average computation, thus we can ignore very rare corner-cases (aka outliers) and ultimately save resources.
%