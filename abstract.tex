%%% -*-LaTeX-*-
%%% This is the abstract for the thesis.
%%% It is included in the top-level LaTeX file with
%%%
%%%    \preface    {abstract} {Abstract}
%%%
%%% The first argument is the basename of this file, and the
%%% second is the title for this page, which is thus not
%%% included here.
%%%
%%% The text of this file should be about 350 words or less.
%
Arithmetic in machines unavoidably introduce roundoff errors stemming from the use of finite-precision formats as a proxy for real (ideal) arithmetic. 
%
It is critical, in particular in safety-critical systems, not only to be aware of these errors but also to \emph{measure} them.
%

In \emph{numerical analysis} we study how to bound roundoff errors, and how they propagate through arithmetic expressions.
%

There are several challenges in roundoff error analysis. 
%
First, the magnitude of the roundoff errors depends on the range of the input variables. 
%
Moreover, it is not trivial to study how errors propagate in arbitrary expressions, in particular, those containing non-linearities or transcendental functions.
%
Furthermore, extending roundoff error analysis to computer programs, rather than straight-line expressions, comes with the burden of dealing with programming constructs like conditionals and loops. 
%
Indeed, roundoff errors can alter the ideal control-flow of a program, thus an if-statement might evaluate to true in ideal arithmetic (in reals), but because of roundoff errors, the same if-statement evaluates to false in the machine. 
%
This phenomenon is called instability jump.
%
%In other words, all the computations obey the worst-case error bound.
%
%Worst-case error bounds are priceless in those applications, like safety-critical systems, where we need such strong guarantees.
%
%In other words, no one would feel comfortable traveling on autonomous running aircrafts which are safe "on average".
%
%At the same time, worst-case analysis could be very conservative, and it might overestimate the average (or typical) error by many orders of magnitude. 
%

In this thesis, we study how to bound roundoff errors in computer programs.

First, we show how to use symbolic execution to reason about the mix of floating-point and real arithmetics, and we use the resulting prototype to detect instability jumps (control-flow instabilities) in computer programs.
%

Second, we show how to use worst-case error analysis to design robust micro-controllers, where we include roundoff errors in the set of all the disturbances affecting the system.
%

Finally, we show how to tackle the conservativeness of worst-case error analysis, by introducing our framework to compute probabilistic roundoff errors.