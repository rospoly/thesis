% !TeX root=./main.tex
%%% -*-LaTeX-*-
%%% This is the abstract for the thesis.
%%% It is included in the top-level LaTeX file with
%%%
%%%    \preface    {abstract} {Abstract}
%%%
%%% The first argument is the basename of this file, and the
%%% second is the title for this page, which is thus not
%%% included here.
%%%
%%% The text of this file should be about 350 words or less.
%

Arithmetic in computers unavoidably introduces roundoff errors stemming from the use of finite-precision formats to represent real numbers.
%
It is essential, for example in safety-critical applications, not only to be aware of the existence of these errors, but also to be able to analyze them.

%
%bound them using rigorous analysis techniques.
%
%which comes with several challenges. 
%
% The aim of this thesis is the rigorous analysis of roundoff errors in floating-point programs.
% With which comes several challenges.

%
The magnitude of roundoff errors depends primarily on the range of the input variables involved in the computations.
%
It is not trivial to study how errors propagate through computations, in particular, when dealing with non-linearities or transcendental functions.
%
Furthermore, extending roundoff error analysis to computer programs, rather than straight-line expressions, comes with the burden of dealing with common programming constructs, like conditionals, where roundoff errors can alter the ideal execution of a program. 
%
Hence, an if-statement conditional might evaluate to true in reals, but because of roundoff errors, the same conditional may evaluate to false in the computer. 
%
%In other words, all the computations obey the worst-case error bound.
%
%Worst-case error bounds are priceless in those applications, like safety-critical systems, where we need such strong guarantees.
%
%In other words, no one would feel comfortable traveling on autonomous running aircrafts which are safe "on average".
%
%At the same time, worst-case analysis could be very conservative, and it might overestimate the average (or typical) error by many orders of magnitude. 
%

The main contributions of this thesis improve the state-of-the-art in the area of rigorous analysis of roundoff errors in floating-point programs.
%
First, we show how to combine automated theorem provers and error analysis techniques to reason about the mix of floating-point and real arithmetics.
%
We use the resulting prototype to detect control-flow instabilities in floating-point programs.
%
%1  What we do?
Second, we show how to embed rigorous worst-case error analysis techniques in the design of micro-controllers.
%
This allows us to sensibly reduce the memory foot-print of the controller, and address the roundoff error stemming from the low-cost floating-point precision format as a disturbance affecting the system, without impacting the stability of the system.
%
Finally, we show the limitations of worst-case error analysis when dealing with probabilistic computations. 
%
We introduce a rigorous probabilistic framework to compute average roundoff errors that addresses the identified limitations.
%
Our prototype implementation provides to the user much more useful probabilistic error bounds compared to the state-of-the-art worst-case errors.