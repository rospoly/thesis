\documentclass[runningheads,american,orivec,fleqn]{llncs}
%

\usepackage{amsfonts}
\usepackage{amsmath}
\usepackage{mathtools}[2011/02/12]
\usepackage{mdwlist}
\usepackage{nicefrac}
\usepackage{stmaryrd}
\usepackage{tikz}
\usepackage{multirow}
\usepackage[left=1.228in, right=1.228in, top=1.024in, bottom=1.187in]{geometry}

\usepackage{etoolbox} % for "\patchcmd" macro
\makeatletter
\patchcmd{\ps@headings}{\rlap{\thepage}}{}{}{}
\patchcmd{\ps@headings}{\llap{\thepage}}{}{}{}
\makeatother
\pagestyle{headings} % reload the now-modified "headings" page style

\PassOptionsToPackage{hyphens}{url}\usepackage[linkcolor=blue, urlcolor=blue, citecolor=blue, backref=false, final]{hyperref}
\usepackage[fancyproofs,fancyexamples,squareitemtag]{theorems}[2015/07/09]%,final,extended

% internals 

\providecommand*{\wgmWarn}[1]{\typeout{*** using WgMacros def of \string#1}}
\providecommand*{\ProvideWarnCommand}[2]{\providecommand*{#1}{\wgmWarn{#1}#2}}

\providecommand*{\backcompatibility}[2]{
  \newcommand*{#1}{\Fbox{USE \texttt{\string#2}}#2}
}

%%%%%%%%%%%%%%%%%%%%%%%%%%%%%%%%%%%%%%%%
%                                      %
%   Support (General Purpose) Macros   %
%                                      %
%%%%%%%%%%%%%%%%%%%%%%%%%%%%%%%%%%%%%%%%

\providecommand*{\parensmathoper}[3][]{\ensuremath{\mathoper{#2}\ifempty{#3}{}{#1(#3#1)}}}
\providecommand*{\parensmathoperEXT}[4][]{\ensuremath{\mathoper{#2}\ifempty{#3}{}{#1(#4#1)}}}
\providecommand*{\syntaxoper}[3]{\mathoper{#1}\BBrackets[#3]{#2}}
\providecommand*{\syntaxoperEXT}[4]{\mathoper{#2}\ifempty{#1}{}{\llbracket#3\rrbracket_{#4}}}%

\providecommand*{\BBrackets}[2][]{\ifempty{#2}{}{\llbracket#2\rrbracket_{#1}}}
\providecommand*{\Braces}[2][]{\ifempty{#2}{}{( #2 )_{#1}}}%\big\big
\providecommand*{\extraArg}[2]{\ifempty{#1}{#2}{\ifempty{#2}{#1}{#1 ,\, #2}}}
\providecommand*{\extraArgSc}[2]{\ifempty{#2}{#1}{\ifempty{#1}{#2}{#1 ;\, #2}}}

\providecommand*{\monobioperator}[3]{
  \newcommand*{#1}[2]{{\ifempty{##2}{#3##1}{##1#2##2}}}
}

%%%%%%%%%%%%%%%%%%%%%%%%%%%%%%%%%%%%%%%%%%%%%%
%                                            %
%   Support Macros defined by theorems.sty   %
%                                            %
%%%%%%%%%%%%%%%%%%%%%%%%%%%%%%%%%%%%%%%%%%%%%%

\providecommand*{\newsmartprefix}[2]{\wgmWarn{\newsmartprefix}}
\providecommand*{\newsmartsprefix}[2]{\wgmWarn{\newsmartsprefix}}
\providecommand*{\newdefinition}[1]{\wgmWarn{\newdefinition}\newdefinitionaux}
\providecommand*{\newdefinitionaux}[2][]{}
%%%%%%%%%%%%%%%%%%%%%%%%%%
%                        %
%   DOMAINS and orders   %
%                        %
%%%%%%%%%%%%%%%%%%%%%%%%%%

% Generic Concrete Domain (for preliminaries)

\newcommand*{\CC}{\mathbb{C}}
\newcommand*{\CCleq}{\sqsubseteq} %% order on Concrete Domain
\newcommand*{\CCnleq}{\nsqsubseteq} %% negated order on Concrete Domain
\monobioperator{\CClub}{\sqcup}{\bigsqcup}
\monobioperator{\CCglb}{\sqcap}{\bigsqcap}
\newcommand*{\CCbot}{\bot} %% bottom of Concrete Domain
\newcommand*{\CCtop}{\top} %% top of Concrete Domain

\newcommand*{\CCcpo}{\cpo{\CC}{\CCleq}}
\newcommand*{\CClattice}{\lattice{\CC}{\CCleq}{\CClub{}{}}{\CCglb{}{}}{\CCtop}{\CCbot}}

% Default Concrete Domain (for paper body, to be localized)

\newcommand*{\Clubbigsym}{\bigsqcup}
\newcommand*{\Cglbbigsym}{\bigsqcap}
\newcommand*{\Clubsym}{\sqcup}
\newcommand*{\Cglbsym}{\sqcap}

\newcommand*{\C}{\CC}
\newcommand*{\Cleq}{\CCleq}
\newcommand*{\Cgeq}{\sqsupseteq}
\newcommand*{\Cnleq}{\CCnleq}
\monobioperator{\Club}{\sqcup}{\bigsqcup}
\monobioperator{\Cglb}{\Cglbsym}{\Cglbbigsym}
\newcommand*{\Cbot}{\CCbot}
\newcommand*{\Ctop}{\CCtop}

\newcommand*{\Ccpo}{\cpo{\C}{\Cleq}}
\newcommand*{\Clattice}{\lattice{\C}{\Cleq}{\Club{}{}}{\Cglb{}{}}{\Ctop}{\Cbot}}

% Generic Abstract Domain (for preliminaries)

\renewcommand*{\AA}{\mathbb{A}}
\newcommand*{\AAleq}{\leq} %% order on Abstract Domain
\newcommand*{\AAgeq}{\geq} %% order on Abstract Domain
\newcommand*{\AAnleq}{\nleq} %% order on Abstract Domain
\monobioperator{\AAlub}{\vee}{\bigvee}
\monobioperator{\AAglb}{\wedge}{\bigwedge}
\newcommand*{\AAbot}{\bot} %% bottom of Abstract Domain
\newcommand*{\AAtop}{\top} %% top of Abstract Domain

\newcommand*{\AAcpo}{\cpo{\AA}{\AAleq}}
\newcommand*{\AAlattice}{\lattice{\AA}{\AAleq}{\AAlub{}{}}{\AAglb{}{}}{\AAtop}{\AAbot}}


% Default Concrete Domain (for paper body, to be localized)

\newcommand*{\Alubbigsym}{\bigvee}
\newcommand*{\Aglbbigsym}{\bigwedge}
\newcommand*{\Alubsym}{\vee}
\newcommand*{\Aglbsym}{\wedge}

\newcommand*{\A}{\AA}
\newcommand*{\Aleq}{\AAleq}
\newcommand*{\Ageq}{\AAgeq}
\newcommand*{\Anleq}{\AAnleq}
\monobioperator{\Alub}{\Alubsym}{\Alubbigsym}
\monobioperator{\Aglb}{\Aglbsym}{\Aglbbigsym}
\newcommand*{\Abot}{\AAbot}
\newcommand*{\Atop}{\AAtop}

\newcommand*{\Acpo}{\cpo{\A}{\Aleq}}
\newcommand*{\Alattice}{\lattice{\A}{\Aleq}{\Alub{}{}}{\Aglb{}{}}{\Atop}{\Abot}}

\newcommand*{\al}{\parensmathoper{\alpha}}
\newcommand*{\ga}{\parensmathoper{\gamma}}

%%%%%%%%%%%%%%%%%%%%%%%%%%%%%%%%%
%                               %
%   Names of Standard Domains   %
%                               %
%%%%%%%%%%%%%%%%%%%%%%%%%%%%%%%%%

\newcommand*{\POS}{\ensuremath{\mathit{POS}}}

\newcommand*{\POSpreabs}[1][]{\parensmathoper{\Gamma_{#1}}}
\newcommand*{\POSabs}{\parensmathoper{\alpha_{\Gamma}}}
\newcommand*{\POSconc}{\parensmathoper{\gamma_{\Gamma}}}
\newcommand*{\POSlub}{\Alub}
\newcommand*{\POSglb}{\Aglb}
\newcommand*{\POSleq}{\Aleq}
\newcommand*{\POSrestrict}[2]{{#1}|_{#2}}
\newcommand*{\POSTp}{\Tp[g]}
\newcommand*{\POSinterp}{\I[g]}
\newcommand*{\POSeval}{\eval[g]}

\newcommand*{\Iff}{\leftrightarrow}

% still to be organized

\newcommand*{\KCTerms}[1][\KVsyms]{\SigTerms{\Csyms}{\Vsyms\cup #1}} %[1][\Fsyms]{\TermsSym(#1, )} %% terms
\newcommand*{\KI}{{\I[\kappa]}}
\newcommand*{\KSz}{\Sz[\kappa]}
\newcommand*{\KTerms}[1][\KVsyms]{\Terms[{\Vsyms\cup #1}]} %[1][\Fsyms]{\TermsSym(#1, )} %% terms
\newcommand*{\KTp}{\Tp[\kappa]}
\newcommand*{\KVsyms}{\widehat{\Vsyms}}
\newcommand*{\Kcut}[2][k]{#2 \lightning_{#1}}
\newcommand*{\Keval}{\eval[\kappa]}
\newcommand*{\depthk}[1][k]{\ensuremath{\mathit{depth}(#1)}}


\newcommand*{\wide}[2]{\ifempty{#1}{\medtriangledown}{#1 \medtriangledown #2}}
\newcommand*{\Lat}{\mathbb{L}}

% BACKWARD COMPATIBILITY

% \backcompatibility{\botAA}{\AAbot}
% \backcompatibility{\botA}{\Abot}
% \backcompatibility{\botCC}{\CCbot}
% \backcompatibility{\botC}{\Cbot}
% \backcompatibility{\cpoAA}{\AAcpo}
% \backcompatibility{\cpoA}{\AAcpo}
% \backcompatibility{\cpoCC}{\CCcpo}
% \backcompatibility{\cpoC}{\CCcpo}
% \backcompatibility{\glbAA}{\AAglb}
% \backcompatibility{\glbA}{\Aglb}
% \backcompatibility{\glbCC}{\CCglb}
% \backcompatibility{\glbC}{\Cglb}
% \backcompatibility{\latticeAA}{\AAlattice}
% \backcompatibility{\latticeA}{\Alattice}
% \backcompatibility{\latticeCC}{\CClattice}
% \backcompatibility{\latticeC}{\Clattice}
% \backcompatibility{\leqAA}{\AAleq}
% \backcompatibility{\leqA}{\Aleq}
% \backcompatibility{\leqCC}{\CCleq}
% \backcompatibility{\leqC}{\Cleq}
% \backcompatibility{\lubAA}{\AAlub}
% \backcompatibility{\lubA}{\Alub}
% \backcompatibility{\lubCC}{\CClub}
% \backcompatibility{\lubC}{\Club}
% \backcompatibility{\nleqAA}{\AAnleq}
% \backcompatibility{\nleqA}{\Anleq}
% \backcompatibility{\nleqCC}{\CCnleq}
% \backcompatibility{\nleqC}{\Cnleq}
% \backcompatibility{\topAA}{\AAtop}
% \backcompatibility{\topA}{\Atop}
% \backcompatibility{\topCC}{\CCtop}
% \backcompatibility{\topC}{\Ctop}

\newcommand*{\dfn}{\coloneq}
\newcommand*{\revdfn}{\eqqcolon}

% \makeatletter
% \@ifpackagelater{mathtools}{2011/02/12}{%
% \DeclarePairedDelimiterX{\set}[2]{\{}{\}}{#1\ifempty{#2}{}{\,\delimsize\vert\, #2}}
% \DeclarePairedDelimiterX{\mset}[2]{\langle}{\rangle}{#1\ifempty{#2}{}{\,\delimsize\vert\, #2}}
% }{%
% \PackageWarning{WGmacros}{The version of `mathtools' package is too old, expect wrong output}
% \DeclarePairedDelimiter{\set}{\{}{|}
% \DeclarePairedDelimiter{\mset}{\langle}{|}
% }
% \makeatother

\providecommand{\annote}[1]{}

\newcommand*{\cpo}[2]{{(#1, \, \mathord{#2})}}

\newcommand*{\lattice}[6]{{(#1, \, \mathord{#2}, \, \annote{$\vee$}\mathord{#3}, 
  \, \mathord{#4}\ifempty{#5}{}{, \,
  \annote{$\top$}\mathord{#5}, \, \mathord{#6}})}}

\newcommand*{\semilattice}[5]{{(#1,
                                \, \mathord{#2},
                                \, \mathord{#3},
                                \, \mathord{#4},
                                \, \mathord{#5})}}

\newcommand*{\zseq}[2][]{\overrightarrow{{#2}_{#1}}} %% indexed (short) sequence
\newcommand*{\seq}[2][n]{#2_{1}, \dots, #2_{#1}} %% indexed sequence
\newcommand*{\hoseq}[3][n]{#2{{#3}_{1}}, \dots, #2{{#3}_{#1}}}
\newcommand*{\emptyseq}{\diamondsuit}

\newcommand*{\glb}{\ensuremath{\mathit{glb}}}
\newcommand*{\lub}{\ensuremath{\mathit{lub}}}

\newcommand*{\quotient}[2]{{#1}\big/_{\! #2}}
\newcommand*{\eqClass}[2]{[{#1}]_{#2}}

\newcommand*{\PowerSet}[1]{\mathop{\powerset}(#1)}
\newcommand*{\FinPowerSet}[1]{\mathop{\powerset_{f}}(#1)}

%%%%%%%%%%%%%%%%%%%%%%%%%%%%%%%%%%%%%%%%%%%%%%%%%%%
%                                                 %
%   Macros that some package may already provide  %
%                                                 %
%%%%%%%%%%%%%%%%%%%%%%%%%%%%%%%%%%%%%%%%%%%%%%%%%%%

\providecommand*{\mathoper}[1]{\mathop{\mathit{#1}}\nolimits}

\providecommand*{\pair}[2]{{\langle #1, \, #2 \rangle}}
\providecommand*{\triple}[3]{{\langle #1, \, #2, \, #3 \rangle}}
\providecommand*{\quadruple}[4]{{\langle #1, \, #2, \, #3, \, #4 \rangle}}

\providecommand*{\true}{\mathit{true}}       %% simbolo di costante logica (T)
\providecommand*{\false}{\mathit{false}}     %% simbolo di costante logica (F)
\providecommand*{\id}[1][]{\mathit{id}_{#1}} %% funzione identica

\providecommand*{\True}{\mathit{True}}
\providecommand*{\False}{\mathit{False}}

\providecommand*{\nimplies}{\nRightarrow}
\renewcommand*{\implies}{\Rightarrow}

%%%%%%%%%%%%%%%%%%%%%%%%%%%%%%%%%%%%%%%%%%%%%%%%%%%%%%%%%%%%%%%%%%%
%                                                                 %
%   Symbols that some package like MnSymbol may already provide   %
%                                                                 %
%%%%%%%%%%%%%%%%%%%%%%%%%%%%%%%%%%%%%%%%%%%%%%%%%%%%%%%%%%%%%%%%%%%

\ProvideWarnCommand{\powerset}{\wp} 
\ProvideWarnCommand{\llbracket}{[\![}
\ProvideWarnCommand{\rrbracket}{]\!]}

\ProvideWarnCommand{\coloneqq}{\mathrel{\vcentcolon=}} %% :=
\ProvideWarnCommand{\coloneq}{\mathrel{\coloneqq}} 
\ProvideWarnCommand{\eqqcolon}{\mathrel{=\vcentcolon}} % =:

% Negated symbols
\ProvideWarnCommand{\ngg}{\not\gg} 
\ProvideWarnCommand{\nin}{\notin}
\ProvideWarnCommand{\notin}{\not\in}
\ProvideWarnCommand{\npreccurlyeq}{\not\preccurlyeq}
\ProvideWarnCommand{\nsimeq}{\not\simeq}
\ProvideWarnCommand{\nsqsubseteq}{\not\sqsubseteq}
\ProvideWarnCommand{\nrightarrowtriangle}{\not\rightarrowtriangle}
\ProvideWarnCommand{\nRightarrow}{\not\Rightarrow} 

% BACKWARD COMPATIBILITY

\backcompatibility{\equivClass}{\quotient}

\makeatletter
\@ifpackageloaded{listings}{
    \newcommand*{\mlst}[1]{\mbox{\lstinline[multicols=]`#1`}}
}{
    \newcommand*{\mlst}[1]{\fbox{\fbox{load package listings}\texttt{\string#1}}}
}

\@ifpackageloaded{datetime2}{
    \newcommand*{\currenttime}{\DTMcurrenttime}
}{
}
\makeatother

\providecommand*{\currenttime}{\fbox{NO TIME package}}
\providecommand*{\mathscr}{\PackageWarning{WGmacros}{no mathscr font}\mathcal} 
\providecommand*{\mathfrak}{\PackageWarning{WGmacros}{no mathfrak font}\mathcal} 

\providecommand*{\eg}   {e.g.} 
\providecommand*{\etc}  {\emph{etc}.}
\providecommand*{\ie}   {i.e.,} 
\providecommand*{\resp} {respectively}
\providecommand*{\st}   {s.t.}
\providecommand*{\wrt}  {w.r.t.}
\providecommand*{\wlg}  {w.l.o.g.}
% \providecommand*{\Wlog} {\wlg}
% 
\newcommand*{\paradigm}[1][] {\Fbox{default #1 paradigm}}
\newcommand*{\proglang}{\Fbox{default programming language}}
\newcommand*{\program}[1][A]{\ifempty{#1}{P}{p}rogram}
\newcommand*{\programs}[1][A]{\program[#1]s}
\newcommand*{\progequiv}{\program\ equivalence}
\newcommand*{\progequivs}{\progequiv s}
\newcommand*{\progrule}{\program\ rule}
\newcommand*{\progrules}{\progrule s}
\newcommand*{\query}[1][A]{\ifempty{#1}{Q}{q}uery}
\newcommand*{\queries}[1][A]{\ifempty{#1}{Q}{q}ueries}
\newcommand*{\expression}[1][A]{\ifempty{#1}{E}{e}xpression}
\newcommand*{\expressions}[1][A]{\expression[#1]s}
\newcommand*{\deffun}[1][A]{\ifempty{#1}{defined }{}function}
\newcommand*{\deffuns}[1][A]{\deffun[#1]s}


\providecommand*{\bibURLext}[2]{\draft{CONSIDER convert bib entry with `url' field!}Available at URL: \href{#1}{\texttt{#2}}}
\providecommand*{\bibURL}[1]{\draft{CONSIDER convert bib entry with `url' field!}Available at URL: \href{#1}{\texttt{#1}}}
\providecommand*{\href}[2]{\wgmWarn{\href}\underline{#2}}

\providecommand*{\email}[1]{\texttt{#1}}

\newcommand*{\xcite}[1]{[{#1}]}

%%%%%%%%%%%%%%%%%%%%%
%%%%%%%%%%%%%%%%%%%%%
\RequirePackage{color}

\newcommand*{\Fbox}{\fcolorbox{black}{yellow}}
\newcommand*{\Flbox}{\fcolorbox{black}{lightyellow}}
\newcommand*{\Dbox}[1]{\draft{\Fbox{#1}}}

\definecolor{aliciacolor}{RGB}{0,125,0}

\newcommand{\Cesar}[1]{\textcolor{red}{[Cesar's comment: #1]}}
\newcommand{\Santi}[2][]{\hole{\colorbox{yellow}{Santi} #1 \quad #2}}
\newcommand{\Marco}[2][]{{\color{blue}\hole{\colorbox{yellow}{Marco} #1 \quad #2}}}
\newcommand{\MarcoPN}[2]{\begin{draftpleasenote}[Marco: #1]\color{blue} #2\end{draftpleasenote}}
\newcommand{\Maria}[2][]{\hole{\colorbox{yellow}{Maria} #1 \quad #2}}
\newcommand{\Moreno}[2][]{\hole{\colorbox{yellow}{Moreno} #1 \quad #2}}

\newcommand{\Alicia}[2][]{\hole{\textbf{#1}\hfill{\color{aliciacolor}\textbf{Alicia}}\\[-1.7ex]\rule{\linewidth}{0.5mm}\\ #2}}
\newcommand{\Giovanni}[2][]{\hole{\textbf{#1}\hfill{\color{blue}\textbf{Giovanni}}\\[-1.7ex]\rule{\linewidth}{0.5mm}\\ #2}}

% nonstandard color names

\RequirePackage{xcolor}

\newcommand{\FromTo}[2]{\csname#1\endcsname[\colorbox{pink}{@#2}]}

\newcommand{\Laura}[2][]{\hole{\textbf{#1}\hfill{\color{violet}\textbf{Laura}}\\[-1.7ex]\rule{\linewidth}{0.5mm}\\ #2}}
\newcommand{\Luca}[2][]{\hole{\textbf{#1}\hfill{\color{orange}\textbf{Luca}}\\[-1.7ex]\rule{\linewidth}{0.5mm}\\ #2}}
\newcommand{\MarcoA}[2][]{\hole{\textbf{#1}\hfill{\color{orange}\textbf{MarcoA}}\\[-1.7ex]\rule{\linewidth}{0.5mm}\\ #2}}

% \newcommand{\AtAlicia}{\FromTo{Marco}{Alicia}}%[1]\Marco[\colorbox{pink}{@Alicia}]{#1}}
% \newcommand{\AtGiovanni}{\FromTo{Marco}{Giovanni}}%[1]{\Marco[\colorbox{pink}{@Giovanni}]{#1}}
% \newcommand{\AtLaura}{\FromTo{Marco}{Laura}}%[1]{\Marco[\colorbox{pink}{@Laura}]{#1}}
% \newcommand{\AtLuca}{\FromTo{Marco}{Luca}}%[1]{\Marco[\colorbox{pink}{@Luca}]{#1}}


\makeatletter
\providecommand{\ifempty}[3]{\def\@@@temp{#1}\ifx\@@@temp\@empty#2\else#3\fi}
\@ifpackageloaded{mathtools}{}{\PackageWarning{WGmacros}%
{Please load `mathtools' package explicitely in preamble}
\RequirePackage{mathtools}}
\makeatother

%%%%%%%%%%%%%%%%%%%%%%%%%%
%                        %
%   DOMAINS and orders   %
%                        %
%%%%%%%%%%%%%%%%%%%%%%%%%%

\newcommand*{\Cinterps}[1][]{\mathbb{I}_{#1}} %% interpretations
\newcommand*{\Ainterps}[1][\A]{\mathbb{I}_{#1}} %% interpretations
\newcommand*{\CIleq}{\Cleq} %% order on interpretetions
\newcommand*{\AIleq}{\Aleq} %% order on interpretetions
\newcommand*{\CInleq}{\Cnleq} %% order on interpretetions
\newcommand*{\AInleq}{\Anleq} %% order on interpretetions

%%%%%%%%%%%%%%%%%%%%%%%%%%%%%%%%%%%%%%%%%
%                                       %
%   TERMS (signature, positions, ...)   %
%                                       %
%%%%%%%%%%%%%%%%%%%%%%%%%%%%%%%%%%%%%%%%%


% private (just for redefinition)
\newcommand*{\mathT}{\mathit}
\newcommand*{\TermsSym}{\mathT{T}}
\newcommand*{\LinTermsSym}{\mathT{L}{\scriptstyle\TermsSym}}
\newcommand*{\twoParsDenot}[3]{\mathop{#1}(#2, #3)}
% public

\newcommand*{\SigTerms}[2]{\twoParsDenot{\TermsSym}{#1}{#2}}
\newcommand*{\LinSigTerms}[2]{\twoParsDenot{\LinTermsSym}{#1}{#2}}

\newcommand*{\Osyms}{\Omega}
\newcommand*{\FOsyms}{\widetilde\Osyms}       %% predicate symbols (signature)
\newcommand*{\Psyms}{\Pi}       %% predicate symbols (signature)
\newcommand*{\Fsyms}{\Sigma}    %% function symbols (signature)
\newcommand*{\Csyms}{\mathT{C}} %% constructor symbols
\newcommand*{\Dsyms}{\mathT{D}} %% defined symbols
\newcommand*{\Vsyms}{\mathT{V}} %% variable symbols
\backcompatibility{\Fsym}{\Fsyms}
\backcompatibility{\Csym}{\Csyms}
\backcompatibility{\Dsym}{\Dsyms}
\backcompatibility{\Vsym}{\Vsyms}




%% terms
\newcommand*{\Terms}[1][\Vsyms]{\SigTerms{\Fsyms}{#1}}
%% constructor terms
\newcommand*{\CTerms}[1][\Vsyms]{\SigTerms{\Csyms}{#1}}

%% Linear terms
\newcommand*{\LinTerms}[1][\Vsyms]{\LinSigTerms{\Fsyms}{#1}}
%% Linear constructor terms
\newcommand*{\LinCTerms}[1][\Vsyms]{\LinSigTerms{\Csyms}{#1}} 

\newcommand*{\leqT}{\preceq} %% order on terms

% \newcommand*{\leqP}{\fbox{change}\preceq} %% order on positions
% \newcommand*{\pos}[2][]{\mathit{O}_{#1}(#2)} %% term positions set 
\newcommand*{\append}{\cdot} %% append
\newcommand*{\replace}[3]{#1[#2]_{#3}} %% term replacement
\newcommand*{\rootPos}{\Lambda} %% root position
\newcommand*{\subterm}[2]{#1|_{#2}} %% subterm

\newcommand*{\vars}[1]{\parensmathoper{var}{#1}} %% term variables
\newcommand*{\fresh}{\ll} %% fresh variant

\newcommand*{\termres}[2]{{#1} \ifempty{#2}{}{\triangleright {#2}}}

\newcommand*{\call}[3]{\termres{\mathit{#1}\ifempty{#2}{}{\mathit{(#2)}}}{#3}}
\newcommand*{\mgc}[4][]{\call{#2}{\mbox{$\zseq[#1]{#3}$}}{#4}}

%% program (rewrite) rule
\newcommand*{\genrule}[2]{#1 \rightarrow #2}
\newcommand*{\prule}[3]{\genrule{\mathit{#1}\ifempty{#2}{}{\mathit{(#2)}}}{#3}}

\newcommand*{\mghead}[3][]{\mathit{#2}\ifempty{#3}{}{(\zseq[#1]{\mathit{#3}})}}
\newcommand*{\mgrule}[4][]{\genrule{\mghead[#1]{#2}{#3}}{#4}}

% \newcommand*{\letop}[2]{\mathop{\mathit{let}} #1 \mathrel{\mathit{in}} #2 } %% let constr. (abstract syntax)

%%%%%%%%%%%%%%%%%%%%%
%                   %
%   SUBSTITUTIONS   %
%                   %
%%%%%%%%%%%%%%%%%%%%%

\newcommand*{\Substs}{\mathit{Substs}}
\newcommand*{\CSubsts}{\Csyms\Substs} %% constructor substitutions
\newcommand*{\MGCsym}{\mathbb{MGC}}
\newcommand*{\MGC}[1][]{\MGCsym_{#1}} %% most general call set

\newcommand*{\leqS}{\preceq} %% order on substitutions
\newcommand*{\gtS}{\succ}    %% order on substitutions
\newcommand*{\eqS}{\simeq}    %% order on substitutions
\newcommand*{\neqS}{\nsimeq}    %% order on substitutions
% \newcommand*{\lubS}{\mathbin{\uparrow}} % oppure mathrel?
\monobioperator{\lubS}{\mathbin{\uparrow}}{\mathbin{\uparrow}}

\newcommand*{\emptysubst}{\varepsilon} %% empty substitution
\newcommand*{\dom}{\parensmathoper{dom}} %% domain
\newcommand*{\range}{\parensmathoper{range}} %% range
\newcommand*{\img}{\parensmathoper{img}}
\backcompatibility{\domrestriction}{\restrict}
\newcommand*{\restrict}[2]{{#1}\mathord{\restriction}_{#2}} %% domain restriction
\newcommand*{\restrictVars}[2]{\restrict{#1}{\vars{#2}}}

%%%%%%%%%%%%%%%%
%              %
%   PROGRAMS   %
%              %
%%%%%%%%%%%%%%%%

\newcommand*{\ProgsSym}{\mathbb{P}}
\newcommand*{\Progs}[1][\Fsyms]{\ProgsSym_{#1}} % \mathit{Progs}

\newcommand*{\ProgSym}{P}
\newcommand*{\prog}[1][]{\ensuremath{\ProgSym_{\mathit{#1}}^{}}} %% generic program
\renewcommand*{\P}{\prog} %% generic program Alias

\newcommand*{\CAnsw}[2]{{#1 \cdot #2}}%\talloblong\mathrel{}
\backcompatibility{\cansw}{\CAnsw}
\newcommand*{\CAleq}{\preceq}

%%%%%%%%%%%%%%%%%%%%%%%%%%%%%%%
%                             %
%   REWRITING AND NARROWING   %
%                             %
%%%%%%%%%%%%%%%%%%%%%%%%%%%%%%%

% \RequirePackage{theorems}
\providecommand*{\starred}[2][*] {\wgmWarn{\starred}\mathrel{\mathord{#2}{}^{#1}}}


\newcommand*{\rewriteStep}[2][\genrule{l}{r}]{\xrightarrow[{#1}]{#2}}
% \newcommand*{\rewriteStep}[2][\genrule{l}{r}]{\xrightarrow[{\scriptscriptstyle #1}]{\scriptscriptstyle #2}}
\newcommand*{\rewrites}[1][]{\rewriteStep[{#1}]{}}
\newcommand*{\rewritesTo}[1][]{\starred{\rewrites[{#1}]}}
\newcommand*{\nrewrites}{\nrightarrow}
\newcommand*{\nrewritesTo}{\starred{\nrewrites}}


\newcommand*{\narrows}[2][]{\mathrel{\mathop{\leadsto}\limits^{\scriptscriptstyle #2}_{\scriptscriptstyle #1}}}
\newcommand*{\narrowsTo}[2][]{\starred{\narrows[#1]{#2}}}
\newcommand*{\narrowStep}[3][\genrule{l}{r}]{\narrows[{#1}]{\extraArg{#2}{#3}}}
\newcommand*{\nnarrows}{\not\leadsto}

%%%%%%%%%%%%%%%%%%%%%%%%%%%%%%%%%%%%%%%%%%
%                                        %
%   SEMANTICS INTERPRETATION FUNCTIONS   %
%                                        %
%%%%%%%%%%%%%%%%%%%%%%%%%%%%%%%%%%%%%%%%%%

% private (just for redefinition)
\newcommand*{\mathC}{\mathit}
\newcommand*{\BehaSym}{\mathC{B}}
\newcommand*{\SemSym}{\mathC{S}}
\newcommand*{\TrSym}{\mathC{N}}
\newcommand*{\OsemSym}{\mathC{O}}
\newcommand*{\FSym}{\mathC{F}}
\newcommand*{\TpSym}{\mathC{P}}
\newcommand*{\evalSym}{\mathC{E}}
% public

\newcommand*{\Q}[2]{#1 \mathbin{\scriptstyle\mathit{in}} #2} % _ in _
\newcommand*{\Beha}[2][ss]{\syntaxoper{\BehaSym^{\mathit{#1}}}{#2}{}} %% general behavior macro
\newcommand*{\Sem}[2][\alpha]{\syntaxoper{\SemSym^{#1}}{#2}{}} % a semantics S
\newcommand*{\Tr}[3][]{\syntaxoperEXT{#2}{\TrSym_{#1}}{\Q{#2}{#3}}{}} %% semantics tree in P
% \newcommand*{\CA}[1]{\syntaxoper{\mathit{CA}}{#1}{}} %% computed answers
% \newcommand*{\CAin}[2]{\CA{\Q{#1}{#2}}} %% computed answers in P

\newcommand*{\Osem}[1]{\syntaxoper{\OsemSym}{#1}{}} %% oper lfp characterization
\newcommand*{\F}[2][\alpha]{\syntaxoper{\FSym^{#1}}{#2}{}} %% general lfp macro
\newcommand*{\Tp}[3][\alpha]{\syntaxoper{\TpSym^{#1}}{#2}{{#3}}} %% general fixed-point op macro
\newcommand*{\TpN}[3][\alpha]{\iter{\Tp[{#1}]{#2}{}}{#3}} %% n-th iteration of the lfp conctruction
\newcommand*{\eval}[4][\alpha]{\syntaxoperEXT{#3}{\evalSym^{#1}}{\termres{#3}{#2}}{#4}}
\newcommand*{\evalrules}[3][\alpha]{\syntaxoper{\xi^{#1}}{#2}{#3}}
\newcommand*{\FUNeq}{\cong} %% variance on tree

\newcommand*{\I}[1][\alpha]{\Isym{#1}} %% generic interpretation
\newcommand*{\Sz}[1][\alpha]{\Ssym{#1}} %% generic specification
\newcommand*{\Rz}[1][\alpha]{\Rsym{#1}}
\newcommand*{\Isym}[1]{\mathC{I}^{#1}}
\newcommand*{\Ssym}[1]{\mathC{S}^{#1}}
\newcommand*{\Rsym}[1]{\mathC{X}^{#1}}

%%%%%%%%%%%%%%%%%%%%%%%%%%%%
%                          %
%   PROGRAM EQUIVALENCES   %
%                          %
%%%%%%%%%%%%%%%%%%%%%%%%%%%%

\newcommand*{\ProgEquiv}[3][]{#2 \thickapprox_{#1} #3} % program equivalence
\newcommand*{\nProgEquiv}[3][]{#2 \not\thickapprox_{#1} #3} % negated program equivalence
\newcommand*{\BehaEq}[1][ss]{\ProgEquiv[\mathit{#1}]}
\newcommand*{\SemEq}[1][{\F[]{}}]{\ProgEquiv[{#1}]} %% program small-step operational equivalence


%%%%%%%%%%%%%%%%%%%
%                 %
%   PROOF TREES   %
%                 %
%%%%%%%%%%%%%%%%%%%

\newcommand{\ands}{\fbox{this should not be used, is just to prevent 
    unobserved redefinition of command \string\ands}}
\newcommand*{\sequent}[4][]{\gdef\and{\quad\hfill}
    \gdef\ands{\;\hfill\ldots\hfill\;}\global\setlabel{#1}
    {#1}\frac{\:\:#2\:\:}{\;#3\;}\;\text{\small%
\begin{tabular}{@{}l@{}} #4 \end{tabular}}}
    
\newcommand*{\twsequent}[3][]{\Fbox{OBSOLETE}\sequent[{#1}]{#2 \and #3}}

\newcommand*{\thsequent}[4][]{\Fbox{OBSOLETE}\sequent[{#1}]{#2 \and #3 \and #4}}

\newcommand*{\manysequent}[3][]{\Fbox{OBSOLETE}\sequent[{#1}]{#2 \ands #3}}

%%%%%%%%%%%%%
%           %
%   TREES   %
%           %
%%%%%%%%%%%%%

\DeclarePairedDelimiter{\Theight}{|}{|}
% \newcommand*{\Tleaf}{\parensmathoper{leaf}} %% leaf nodes of a tree
% \newcommand*{\Tsteps}{\parensmathoper{steps}} %% steps set of a tree
% \newcommand*{\Tstep}[4]{\quadruple{#1}{#2}{#3}{#4}} %% a relaxed-step in a tree
% \newcommand*{\Tsubterm}[2]{\left. #1 \right\Downarrow_{#2}} %% induced subterm tree
\newcommand*{\Tnode}[2]{\CAnsw{#1}{#2}} %% node constructor
\newcommand*{\Tedges}{\parensmathoper{edges}}
\newcommand*{\Tleaves}{\parensmathoper{leaves}} %% leaf nodes of a tree
\newcommand*{\Tmaxpaths}{\parensmathoper{maxpaths}}
\newcommand*{\Tnodes}{\parensmathoper{nodes}}
\newcommand*{\Tpaths}{\parensmathoper{paths}}
\newcommand*{\Troot}{\parensmathoper{rt}} %% root node of a tree
\newcommand*{\Tsiblings}{\parensmathoper{siblings}}
\newcommand*{\Tsons}{\parensmathoper{sons}}


%%%%%%%%%%%%%%%%%%%%%
%                   %
%   Miscellaneous   %
%                   %
%%%%%%%%%%%%%%%%%%%%%

\newcommand*{\mgu}{\parensmathoper{mgu}} %% most general unifier

\newcommand*{\lfp}[1][]{\lfpof[{#1}]{}}
\newcommand*{\gfp}[1][]{\gfpof[{#1}]{}}
\newcommand*{\lfpof}[2][]{\mathoper{lfp}_{#1}\Braces{#2}}
\newcommand*{\gfpof}[2][]{\mathoper{gfp}_{#1}\Braces{#2}}

\newcommand*{\downiter}[2]{#1 \mathord{\downarrow} #2} %% iteration macro
\newcommand*{\iter}[2]{#1 \mathord{\uparrow} #2} %% iteration macro
\newcommand*{\sameof}[3]{#1 \mapsto \iter{#2}{#3} (\mathit{#1})}


\newcommand*{\len}{\parensmathoper{length}}
\newcommand*{\prefixtree}{\parensmathoper{prfxtree}}


% BACKWARD COMPATIBILITY

% \backcompatibility{\CSubst}{\CSubsts}
% \backcompatibility{\Subst}{\Substs}
% \backcompatibility{\Symbols}{\Fsyms}
% \backcompatibility{\emptypos}{\rootPos}
% \backcompatibility{\interpA}{\Ainterps}
% \backcompatibility{\interpC}{\Cinterps}
% \backcompatibility{\interp}{\Cinterps}

% \backcompatibility{\leqIa}{\AIleq}
% \backcompatibility{\leqIc}{\CIleq}
% \backcompatibility{\leqI}{\CIleq}
% \backcompatibility{\nleqIa}{\AInleq}
% \backcompatibility{\nleqIc}{\CInleq}
% \backcompatibility{\nleqI}{\CInleq}
% \backcompatibility{\nrew}{\nrewrites}
% \backcompatibility{\rewrule}{\genrule}
\newcommand*{\rews}[2][noOpt]{\Fbox{USE \texttt{\string\rewritesTo}}\Fbox{$#1$}\rewritesTo[{#2}]}
\newcommand*{\rew}[2][noOpt]{\Fbox{USE \texttt{\string\rewrites}}\Fbox{$#1$}\rewrites[{#2}]}

\makeatletter
\@ifpackageloaded{MnSymbol}{}{\PackageWarning{WGmacros}%
{Please load `MnSymbol' package explicitely in preamble}
\RequirePackage{MnSymbol}}
\makeatother

%%%%%%%%%%%%%%%%%%%%%%%%%%%%%%%%%
%                               %
%   overrides of general defs   %
%                               %
%%%%%%%%%%%%%%%%%%%%%%%%%%%%%%%%%

\renewcommand*{\wrt}{with respect to}

\renewcommand*{\MGCsym}{\mathbb{M}} %% most general call set

\renewcommand*{\genrule}[2]{#1 = #2}

% \renewcommand*{\C}[1][]{\mathbb{M}_{#1}}
% \renewcommand*{\Ctop}[1][]{\mathbf{M}_{#1}}
% \renewcommand*{\Cbot}{\emptyset}

% \renewcommand*{\eval}[4][\alpha]{\syntaxoper{\mathit{A}^{#1}}{#3}{{#4}}} %% semantic evaluation
\renewcommand*{\Q}[2]{\ifempty{#2}{#1}{#2 \mathbin{.} #1}} % D.A -> \Q{A}{D}
\renewcommand*{\TpSym}{\mathC{D}}
\renewcommand*{\TrSym}{\mathC{P}} %% semantics tree in P
\renewcommand*{\evalSym}{\mathC{A}}

%\renewcommand*{\Fsyms}{\CSys,\Psyms}
%\renewcommand*{\Csyms}{\Fbox{fix}\CSys} %% constructor symbols
%\renewcommand*{\Dsyms}{\Psyms} %% defined symbols

% \renewcommand*{\ProgSym}{D} %% generic program=set of decls
% Insieme degli insiemi di dichiarazioni: si usa \Progs
% \renewcommand*{\ProgsSym}{\mathbb{D}} %\mathit{Progs}
% \renewcommand*{\Progs}{\ProgsSym^{\Psyms}} % \mathit{Progs}


%%%%%%%%%%%%%%%%%%%%%%%%%%%%%%%
%                             %
%   Subject Specific macros   %
%                             %
%%%%%%%%%%%%%%%%%%%%%%%%%%%%%%%



\newcommand*{\N}{\ensuremath{\mathbb{N}}}
\newcommand*{\R}{\ensuremath{\mathbb{R}}}
\newcommand*{\Z}{\ensuremath{\mathbb{Z}}}
\newcommand*{\Fp}{\ensuremath{\mathbb{F}}}

\newcommand*{\Interv}[1]{\mathit{Int} \ifempty{#1}{}{_{#1}}}

\newcommand*{\D}{\prog} %% generic program Alias

% \newcommand*{\Conds}[1][\CSys]{\Lambda_{#1}}%\mathit{Agents}
% \newcommand*{\ValConds}[1][\CSys]{\Delta_{#1}}%\mathit{Agents}


\newcommand*{\round}{\parensmathoper{\rho}}
\newcommand*{\real}{\parensmathoper{\mathit{\chi_{r}}}}
\newcommand*{\err}{\parensmathoper{\mathit{\chi_{e}}}}
% \newcommand*{\fp}{\parensmathoper{\mathit{\chi_{\mathit{fp}}}}}

\newcommand*{\evalC}[3][\alpha]{\syntaxoper{\bar{\evalSym}^{#1}}{#2}{{#3}}} %compact eval
\newcommand*{\TpC}[3][\alpha]{\syntaxoper{\bar{\TpSym}^{#1}}{#2}{{#3}}}%compact Tp
\newcommand*{\FC}[2][\alpha]{\syntaxoper{\bar{\FSym}^{#1}}{#2}{}}%compact fixpoint

\newcommand*{\Conf}{\mathit{Conf}}% insieme di configurazioni

\newcommand*{\clauseif}{\ensuremath{\mathrel{\mathord{:}\mathord{-}}}}

\newcommand*{\ra}{\rightarrow}
\newcommand*{\nra}{\not\rightarrow}

%% Syntax %%

\newcommand*{\ADSB}{ADS-B} 
\newcommand*{\CPR}{CPR} 

\newcommand*{\FPtoR}{\parensmathoper{{R}}}
% \newcommand*{\RtoS}{\parensmathoper{\mathit{RtoS}}}
% \newcommand*{\RtoD}{\parensmathoper{\mathit{RtoD}}}
\newcommand*{\RtoF}[2][]{\mathit{F}_{#1}\ifempty{#2}{}{(#2)}}
\newcommand*{\FtoR}{\parensmathoper{\mathit{R}}}
%  {}
% \newcommand*{\FtoRA}{\parensmathoper{\mathbf{F}_{A}}}
\newcommand*{\FtoRB}{\parensmathoper{\mathit{R}_{\FBExprDom}}}
\newcommand*{\FtoRA}{\parensmathoper{\mathit{R}_{\FAExprDom}}}



\newcommand*{\ErrExpr}{\mathit{E}}
\newcommand*{\IntExpr}{\mathit{I}}




\newcommand*{\AExpr} {\mathit{A}}
\newcommand*{\BExpr} {\mathit{B}}
\newcommand*{\FAExpr}{\widetilde{\mathit{A}}}
\newcommand*{\FBExpr}{\widetilde{\mathit{B}}}
% \newcommand*{\FAEdom}{\mathit{AExpr}}
% \newcommand*{\FAEdom}{\mathit{BExpr}}
% \newcommand*{\FAEdom}{\widetilde{\mathit{AExpr}}}
% \newcommand*{\FAEdom}{\widetilde{\mathit{BExpr}}}
\newcommand*{\AExprDom} {\mathbb{A}}
\newcommand*{\BExprDom} {\mathbb{B}}
\newcommand*{\FAExprDom}{\widetilde{\mathbb{A}}}
\newcommand*{\FBExprDom}{\widetilde{\mathbb{B}}}


\newcommand*{\leqB}{\mathrel{\Rightarrow}}
\newcommand*{\nleqB}{\mathrel{\not\Rightarrow}}
\newcommand*{\equivB}{\mathrel{\Leftrightarrow}}

\newcommand*{\EExpr} {\mathit{E}}
\newcommand*{\EExprDom}{\mathbb{E}}
\newcommand*{\Einf}{+\infty}


\newcommand*{\StmDom}{\mathbb{S}}
\newcommand*{\Stm}{\mathit{S}}
\newcommand*{\FStmDom}{\mathbb{S}} %{\widetilde{\mathbb{E}}}
\newcommand*{\FStm}{\mathit{S}} % {\widetilde{\mathit{E}}}
\newcommand*{\Decl}{\mathit{Decl}}
\newcommand*{\Prog}{\mathbb{P}}
\newcommand*{\FProg}{\widetilde{\mathbb{P}}}

\newcommand*{\leqCEB}{\mathrel{\leq}}
% \newcommand*{\leCEB}{<}
\newcommand*{\leqE}{\mathrel{\leq}}
\newcommand*{\Vars}{\parensmathoper{\mathit{vars}}}

\newcommand*{\supC}{\mathbf{C}}
\monobioperator{\lubC}{\sqcup}{\bigsqcup}
\monobioperator{\glbC}{\sqcap}{\bigsqcap}
\newcommand*{\leqC}{\mathrel{\sqsubseteq}}
\newcommand*{\topC}{\supC}
\newcommand*{\botC}{\emptyset}
% \newcommand*{\botC}{\{\cerr{\true}{\true}{\emptyset}{0}{\emptyPath}{\stt}, \cerr{\true}{\true}{\emptyset}{0}{\emptyPath}{\unstt} \}}
\newcommand*{\latticeC}{\lattice{\C}{\leqC{}{}}{\lubC{}{}}{\glbC{}{}}{\topC}{\botC}}

\renewcommand*{\A}[1][\paths]{\dot{\mathbb{C}}_{#1}}
\newcommand*{\supA}[1][\paths]{\dot{\mathbf{C}}}
\monobioperator{\lubA}{\mathrel{\dot\sqcup}}{\dot\bigsqcup}
\monobioperator{\glbA}{\mathrel{\dot\sqcap}}{\dot\bigsqcap}
\newcommand*{\leqA}{\mathrel{\dot\sqsubseteq}}
\newcommand*{\topA}{\top_{\A}}
\newcommand*{\botA}{\emptyset}
\newcommand*{\latticeA}{\lattice{\A}{\leqA{}{}}{\lubA{}{}}{\glbA{}{}}{\topA}{\botA}}
\newcommand*{\elemA}{\dot{C}}
\newcommand*{\elema}{\dot{c}}

% meta-variables
\newcommand*{\fpcnst}{\mathit{\tilde{d}}}
\newcommand*{\rcnst}{\mathit{d}}
\newcommand*{\rval}{\mathit{v}}
\newcommand*{\fpval}{\tilde{\mathit{v}}}
\newcommand*{\rx}{\mathit{x}}
\newcommand*{\fpx}{\tilde{\mathit{x}}}

% \newcommand*{\mvFA}{\widetilde{\mathit{A}}}
% \newcommand*{\mvFB}{\widetilde{\mathit{B}}}
% \newcommand*{\mvFE}{\mathit{E}}
% \newcommand*{\mvA}{\mathit{A}}
% \newcommand*{\mvB}{\mathit{B}}
% \newcommand*{\mvE}{\mathit{E}}

\newcommand*{\FExpr}{\FStm}
\newcommand*{\FExprDom}{\FStmDom}

\newcommand{\bexpr}{\phi}
\newcommand{\vraexpr}{\mathit{expr}}
\newcommand{\vfpaexpr}{\widetilde{\vraexpr}}


\newcommand*{\fprog}{P}

\newcommand*{\fpfun}{\tilde{f}}
\newcommand*{\rfun}{f}
\newcommand*{\fpaexpr}{\tilde{A}}
\newcommand*{\stm}{S}


\newcommand*{\rop}{\odot}
\newcommand*{\fop}{\mathrel{\widetilde{\odot}}}

\newcommand*{\unbs}{\mathit{single}}
\newcommand*{\unbd}{\mathit{double}}

\newcommand*{\pvsrule}[3]{\genrule{\mathit{#1}\ifempty{#2}{}{\mathit{(#2)}}}{#3}}
\newcommand*{\type}{\mathit{type}}

\newcommand*{\taglet}{\mathrel{\mathit{let}}}
\newcommand*{\tagin}{\mathrel{\mathit{in}}}
\newcommand*{\tagif}{\mathrel{\mathit{if}}}
\newcommand*{\tagthen}{\mathrel{\mathit{then}}}
\newcommand*{\tagelse}{\mathrel{\mathit{else}}}
\newcommand*{\tagelsif}{\mathrel{\mathit{elsif}}}
\newcommand*{\tagendif}{\mathrel{\mathit{fi}}}
\newcommand*{\letStm}[3]{\taglet #1=#2 \tagin #3}
\newcommand*{\ite}[3]{\tagif #1 \tagthen #2 \ifempty{#3}{}{\tagelse #3}}
\newcommand*{\forit}[4]{\mathit{for}(#1,#2,#3,#4)}
\newcommand*{\apcall}[2]{\mathit{#1}\ifempty{#2}{}{(\mathit{#2})}}
\newcommand*{\stabWarning}{\mathit{warning}}

\newcommand*{\grel}{\mathrel{\diamond}}
\newcommand*{\gbin}{\mathrel{\odot}}
\newcommand*{\gmono}{\parensmathoper{\mathit{f}}}

% \newcommand*{\modA}[2]{\mathit{mod} \ifempty{#1}{}{(#1,#2)}}
% \newcommand*{\powA}[2]{\mathit{pow}(#1,#2)}
\newcommand*{\floorA}[1]{\lfloor #1 \rfloor}
\newcommand*{\absA}[1]{\left| #1 \right|}
\newcommand*{\sqrtA}{\sqrt}
\newcommand*{\sinA}{\parensmathoper{\mathit{sin}}}
\newcommand*{\cosA}{\parensmathoper{\mathit{cos}}}

\newcommand*{\Radd}[2]{\ifempty{#1}{\mathop{+}}{#1 \mathop{\tilde{+}} #2}}
\newcommand*{\Rsub}[2]{\ifempty{#1}{\mathop{-}}{#1 \mathop{\tilde{-}} #2}}
\newcommand*{\Rmul}[2]{\ifempty{#1}{\mathop{*}}{#1 \mathop{\tilde{*}} #2}}
\newcommand*{\Rdiv}[2]{\ifempty{#1}{\mathop{/}}{#1 \mathop{\tilde{/}} #2}}
\newcommand*{\Rmod}[2]{\mathit{mod}\ifempty{#1}{}{(#1, #2)}}
\newcommand*{\Rfloor}[1]{\mathit{floor} \ifempty{#1}{}{(#1)}}
\newcommand*{\Rsqrt}[1]{\mathit{sqrt}\ifempty{#1}{}{(#1)}}
\newcommand*{\Rabs}[1]{\mathit{abs}\ifempty{#1}{}{(#1)}}
\newcommand*{\Rneg}[1]{- \ifempty{#1}{}{#1}}
\newcommand*{\Rsin}[1]{\mathit{sin}\ifempty{#1}{}{(#1)}}
\newcommand*{\Rcos}[1]{\mathit{cos}\ifempty{#1}{}{(#1)}}
\newcommand*{\Rarctan}{\parensmathoper{\mathit{atan}}}
\newcommand*{\Rln}{\parensmathoper{\mathit{ln}}}

\newcommand*{\Fadd}[2]{\ifempty{#1}{\mathop{\tilde{+}}}{#1 \mathop{\tilde{+}} #2}}
\newcommand*{\Fsub}[2]{\ifempty{#1}{\mathop{\tilde{-}}}{#1 \mathop{\tilde{-}} #2}}
\newcommand*{\Fmul}[2]{\ifempty{#1}{\mathop{\tilde{*}}}{#1 \mathop{\tilde{*}} #2}}
\newcommand*{\Fdiv}[2]{\ifempty{#1}{\mathop{\tilde{/}}}{#1 \mathop{\tilde{/}} #2}}
\newcommand*{\Fmod}[2]{\ifempty{#1}{\widetilde{\mathit{mod}}}{\widetilde{\mathit{mod}}(#1, #2)}}
\newcommand*{\Ffloor}[1]{\ifempty{#1}{\widetilde{\mathit{floor}}}{\widetilde{\mathit{floor}}(#1)}}
\newcommand*{\Fsqrt}[1]{\ifempty{#1}{\widetilde{\mathit{sqrt}}}{\widetilde{\mathit{sqrt}}(#1)}}
\newcommand*{\Fabs}[1]{\ifempty{#1}{\widetilde{\mathit{abs}}}{\widetilde{\mathit{abs}}(#1)}}
\newcommand*{\Fneg}[1]{\ifempty{#1}{\tilde{-}}{\tilde{-} #1}}
\newcommand*{\Fsin}[1]{\ifempty{#1}{\widetilde{\mathit{sin}}}{\widetilde{\mathit{sin}}(#1)}}
\newcommand*{\Fcos}[1]{\ifempty{#1}{\widetilde{\mathit{cos}}}{\widetilde{\mathit{cos}}(#1)}}
\newcommand*{\Farctan}{\parensmathoper{\widetilde{\mathit{atan}}}}
\newcommand*{\Fln}{\parensmathoper{\widetilde{\mathit{ln}}}}

\newcommand*{\minp}{\parensmathoper{\mathit{min}}}


\renewcommand*{\I}{I} % {\rho}
\newcommand*{\Var}{\mathbb{V}}%variabili
\newcommand*{\FVar}{\widetilde{\mathbb{V}}}%variabili
\newcommand*{\Env}{\mathit{Env}} % \mathbb{E}}\mathit{Env}
\newcommand*{\EnvA}[1][\paths]{\dot{\mathit{Env}}_{#1}} % \mathbb{E}}\mathit{Env}
\newcommand*{\Interp}{\mathbb{I}} % \mathbb{I}

\newcommand*{\env}{\nu}

\newcommand{\fvar}[1][x]{\tilde{#1}}

\newcommand*{\botEnv}{\bot_{\Env}}
\newcommand*{\botEnvA}{\bot_{\EnvA[]}}
\newcommand*{\botI}{\bot_{\mathit{\Interp}}}

\newcommand*{\botUnstt}{\bot_{\unstt}}

\newcommand*{\Bfalse}{\mathit{false}} 
\newcommand*{\Btrue}{\mathit{true}} 

\newcommand*{\Pt}{\mathit{\Pi}}
\newcommand*{\pt}{\mathit{\pi}}

\newcommand*{\ruleVar}[1]{\mathit{\pi_{\mathit{var}}}\ifempty{#1}{}{(#1)}}
\newcommand*{\ruleNum}[1]{\mathit{\pi_{\mathit{cnst}}}\ifempty{#1}{}{(#1)}}
\newcommand*{\rulef}[2]{\pi_{#1}\ifempty{#2}{}{(#2)}}
\newcommand*{\ruleUnIte}[3]{\pi_{\mathit{un}}\ifempty{#1}{}{(#1,#2,#3)}}

\newcommand*{\botIA}{{\bot}_{\mathit{\InterpA}}}

%% Semantics %%

\newcommand*{\ceb}{conditional error bound} 
\newcommand*{\cebs}{conditional error bounds} 

\newcommand*{\propGuard}[3]{\ifempty{#1}{\Downarrow}{\ifempty{#2}{\Downarrow_{\cond{#1}{#2}}} {{#3}\Downarrow_{\cond{#1}{#2}}}}}

\newcommand*{\ApropGuard}[2]{\ifempty{#1}{\dot{\Downarrow}}{\ifempty{#2}{\dot{\Downarrow}_{#1}} {{#2}\dot{\Downarrow}_{#1}}}}

\newcommand*{\Ii}{\Cinterps}%dominio interpretazione di un prog.

\newcommand*{\ce}[3]{(#1,#2,#3)}
\newcommand*{\cec}[5]{(#1, #2, #3, #4, #5)}
\newcommand*{\cerr}[6]{\langle #1, #2 \rangle_{#6} \twoheadrightarrow (#3, #4)^{#5}}
\newcommand*{\cond}[2]{(#1, #2)}
\newcommand*{\ct}[5]{\langle #1, #2 \rangle_{#5} \twoheadrightarrow (#3, #4)}

\newcommand*{\rcond}{\eta}
\newcommand*{\fpcond}{\tilde{\eta}}
\newcommand*{\rres}{r}
\newcommand*{\fpres}{\tilde{r}}

\newcommand*{\cerrA}[5]{\langle #1 \rangle_{#5} \twoheadrightarrow (#2, #3)^{#4}}

\newcommand*{\stt}{\mathbf{s}}
\newcommand*{\unstt}{\mathbf{u}}


\newcommand*{\Cond}{\mathit{Cond}}

\newcommand*{\Aa}{\eval[]{}} % [2]{#1}{#2}sem agente

\newcommand*{\Asem}[2]{\mathoper{\mathcal{A}}\ifempty{#1}{}{\llbracket#1\rrbracket_{#2}}}
\newcommand*{\Bsem}[2]{\mathoper{\mathcal{B}}\ifempty{#1}{}{\llbracket#1\rrbracket_{#2}}}
\newcommand*{\Ssem}[3]{\mathoper{\mathcal{E}}\ifempty{#1}{}{\llbracket#1\rrbracket_{#2}}}
\newcommand*{\Dsem}[2]{\mathoper{\mathcal{P}}\ifempty{#1}{}{\llbracket#1\rrbracket_{#2}}}
\newcommand*{\Fsem}[1]{\mathoper{\mathcal{F}}\ifempty{#1}{}{\llbracket#1\rrbracket}}
\newcommand*{\Qq}[2]{\Q{#2}{#1}}



%% FP formalization %%

\newcommand*{\emin}{e_{\mathit{min}}}
\newcommand*{\ulp}{\parensmathoper{\mathit{ulp}}}
% real variables
\newcommand*{\rv}[1]{\mathit{r}\ifempty{#1}{}{_{#1}}}
% error values
\newcommand*{\ev}[1]{\mathit{e}\ifempty{#1}{}{_{#1}}}
% floating point variables
\newcommand*{\fpv}[1]{\tilde{\mathit{v}}\ifempty{#1}{}{_{#1}}}
\newcommand*{\Rfpv}[1]{\FPtoR{\tilde{\mathit{v}}\ifempty{#1}{}{_{#1}}}}
% exponent 
\newcommand*{\fpexp}[1]{\ensuremath{\mathit{e}_{#1}}}
\newcommand*{\fpbase}{\beta}
\newcommand*{\rbool}{\phi}
\newcommand*{\fpbool}{\tilde{\phi}}
% floating-point addition
\newcommand*{\infixbinmathoper}[3]{\ensuremath{\ifempty{#2}{\mathoper{#1}}{#2\mathoper{#1}#3}}}
\newcommand*{\fpadd}{\infixbinmathoper{\tilde{+}}}
\newcommand*{\fpdiv}{\infixbinmathoper{\tilde{/}}}
% error bound
\newcommand*{\ebound}[2]{\ensuremath{\epsilon_{#1}\ifempty{#2}{}{(#2)}}}
% \newcommand*{\eubound}[3]{\ensuremath{\varepsilon_{\unstt}\ifempty{#1}{}{(#1,#2,#3)}}}
\newcommand*{\eboundA}[2]{\ensuremath{\dot{\epsilon}_{#1}\ifempty{#2}{}{(#2)}}}

%% Tools %%

\newcommand*{\ACSL}{ACSL}
\newcommand*{\tool}{\precisa}
\newcommand*{\Tool}{\precisa}
\newcommand*{\smttool}{FPRoCK}
\newcommand*{\precisa}{PRECiSA}
\newcommand*{\RangeLab}{RangeLab}
\newcommand*{\Fluctuat}{Fluctuat}
\newcommand*{\Clang}{C}
\newcommand*{\ADA}{ADA}
\newcommand*{\Coq}{Coq}
\newcommand*{\HOLlight}{HOL Light}
\newcommand*{\VCFloat}{VCFloat}
\newcommand*{\FPTaylor}{FPTaylor}
\newcommand*{\Zt}{Z3}
\newcommand*{\CVCf}{CVC4}
\newcommand*{\Mathsat}{Mathsat}
\newcommand*{\Scala}{Scala}
\newcommand*{\SMT}{SMT}
\newcommand*{\Gappa}{Gappa}
\newcommand*{\Caduceus}{Caduceus}
\newcommand*{\FramaC}{FramaC}
\newcommand*{\PVS}{PVS}
\newcommand*{\NASAPVS}{NASA PVS Library}
\newcommand*{\RealToFloat}{Real2Float}
\newcommand*{\FPTuner}{FPTuner}
\newcommand*{\Realizer}{Realizer}
\newcommand*{\Molly}{Molly}
\newcommand*{\Colibri}{Colibri}


%% Galois Connections %%

\monobioperator{\lubES}{\cup_{\EExprDom}}{\bigcup_{\EExprDom}}
\monobioperator{\glbES}{\cap_{\EExprDom}}{\bigcap_{\EExprDom}}
\newcommand*{\leqES}{\mathrel{\subseteq_{\EExprDom}}}
\newcommand*{\topES}{\{+\infty\}}
\newcommand*{\botES}{\{0\}}
\newcommand*{\latticeES}{\lattice{\EExprDom}{\leqES{}{}}{\lubES{{}}{{}}}{\glbES{{}}{{}}}{\topES}{\botES}}

\newcommand*{\AExprDomA}{\dot{\AExprDom}}
\newcommand*{\leqAA}{\mathrel{\dot{\preceq}}}
\newcommand*{\topAA}{\top_{\AExprDomA}}

\newcommand*{\EExprDomA}{\dot{\EExprDom}}
\newcommand*{\leqEA}{\mathrel{\dot{\leq}}}
\newcommand*{\leEA}{\mathrel{\dot{<}}}
\newcommand*{\geqEA}{\mathrel{\dot{\geq}}}
\monobioperator{\lubEA}{\mathrel{\dot{\oplus}}}{\dot{\bigoplus}}
\monobioperator{\glbEA}{\mathrel{\dot{\otimes}}}{\dot{\bigotimes}}
\newcommand*{\topEA}{\top_{\EExprDomA}}
\newcommand*{\botEA}{\bot_{\EExprDomA}}

\newcommand*{\BExprDomA}{\dot{\BExprDom}}
\newcommand*{\leqBA}{\mathrel{\dot{\Rightarrow}}}
\newcommand*{\nleqBA}{\mathrel{\dot{\not\Rightarrow}}}
\monobioperator{\lubBA}{\mathrel{\dot{\vee}}}{\dot{\bigvee}}
\monobioperator{\glbBA}{\mathrel{\dot{\wedge}}}{\dot{\bigwedge}}
\newcommand*{\topBA}{\top_{\BExprDomA}}
\newcommand*{\botBA}{\bot_{\BExprDomA}}

\newcommand*{\leqBS}{\mathrel{\hat{\Rightarrow}}}
\newcommand*{\equivBS}{\mathrel{\hat{\Leftrightarrow}}}
\monobioperator{\lubBS}{\mathrel{\hat{\vee}}}{\hat{\bigvee}}
\monobioperator{\glbBS}{\mathrel{\hat{\wedge}}}{\hat{\bigwedge}}
\newcommand*{\topBS}{\{\cond{\true}{\true}\}}
\newcommand*{\botBS}{\{\cond{\false}{\false}\}}

\newcommand*{\latticeEA}{\lattice{\EExprDomA}{\leqEA{}{}}{\lubEA{{}}{{}}}{\glbEA{{}}{{}}}{\topEA}{\botEA}}
\newcommand*{\latticeBA}{\lattice{\BExprDomA}{\leqBA{}{}}{\lubBA{{}}{{}}}{\glbBA{{}}{{}}}{\topBA}{\botBA}}

\newcommand*{\alphaA}[1]{\alpha_{\AExprDom}\ifempty{#1}{}{(#1)}}
\newcommand*{\gammaA}[1]{\gamma_{\AExprDom}\ifempty{#1}{}{(#1)}}
\newcommand*{\alphaE}[1]{\alpha_{\EExprDom}\ifempty{#1}{}{(#1)}}
\newcommand*{\gammaE}[1]{\gamma_{\EExprDom}\ifempty{#1}{}{(#1)}}
\newcommand*{\alphaB}[1]{\alpha_{\BExprDom}\ifempty{#1}{}{(#1)}}
\newcommand*{\gammaB}[1]{\gamma_{\BExprDom}\ifempty{#1}{}{(#1)}}

\newcommand*{\alphak}[2]{\alpha_{#1}\ifempty{#2}{}{(#2)}}
\newcommand*{\gammak}[2]{\gamma_{#1} \ifempty{#2}{}{(#2)}}
\newcommand*{\alphakInt}[2]{\bar{\alpha}_{#1}\ifempty{#2}{}{(#2)}}
\newcommand*{\gammakInt}[2]{\bar{\gamma}_{#1} \ifempty{#2}{}{(#2)}}
\newcommand*{\alphaU}[1]{\alpha^{\unstt}\ifempty{#1}{}{(#1)}}
\newcommand*{\gammaU}[1]{\gamma^{\unstt}\ifempty{#1}{}{(#1)}}
\newcommand*{\alphaUk}[2]{\alpha^{\unstt}_{#1}\ifempty{#2}{}{(#2)}}

\newcommand*{\mergeA}[2]{\ifempty{#2}{\bigodot \ifempty{#1}{}{#1}}{#1 \odot #2}}

\newcommand*{\InterpC}{\mathbb{I}}
\newcommand*{\InterpA}[1][\pathsInterp]{\dot{\mathbb{I}}_{#1}}

%% Abstract Semantics %%

\newcommand*{\AAsem}[4]{\text{\mathversion{normal2}$\mathoper{\dot{\mathcal{A}}^{{#1}}}$\mathversion{normal}$\ifempty{#2}{}{\llbracket#2\rrbracket^{#4}_{#3}}$}}
\newcommand*{\ASsem}[5]{\text{\mathversion{normal2}$\mathoper{\dot{\mathcal{E}}^{{#1}}}$\mathversion{normal}$\ifempty{#2}{}{\llbracket#2\rrbracket_{(#3,#4)}^{#5}}$}}
\newcommand*{\ADsem}[3]{\text{\mathversion{normal2}$\mathoper{\dot{\mathcal{P}}^{{#1}}}$\mathversion{normal}$\ifempty{#2}{}{\llbracket#2\rrbracket_{#3}}$}}
\newcommand*{\AFsem}[2]{\text{\mathversion{normal2}$\mathoper{\dot{\mathcal{F}}^{{#1}}}$\mathversion{normal}$\ifempty{#2}{}{\llbracket#2\rrbracket}$}}

\newcommand*{\equivA}{\dot{\equiv}}

\newcommand*{\gtrACEB}{\mathrel{\gtrdot}}
\newcommand*{\leqACEB}{\mathrel{\leqdot}}
\newcommand*{\geqACEB}{\mathrel{\geqdot}}
\monobioperator{\lubACEB}{\dot\sqcup}{\dot\bigsqcup}

\newcommand*{\widek}[3]{\ifempty{#2}{\medtriangledown_{#1}}{#2 \medtriangledown_{#1} #3}}

%% Examples %%

\newcommand{\tcpa}{t_{\mbox{\tiny cpa}}}


%% Constraint Systems%%
\newcommand*{\CS}{\langle \CSdom, \CSleq, \CSlub \rangle}
% \newcommand*{\CSys}{\mathbf{C}} %constraint system
\newcommand*{\CSdom}{\mathversion{normal2}\text{$\mathcal{C}$}\mathversion{normal}} %insieme di constraint
\newcommand*{\CSimp}{\mathrel{\vdash}}%implicazione
\newcommand*{\CSnimp}{\mathrel{\nvdash}}%non-implicazione
\newcommand*{\CSequiv}{{\equiv}_{\CSdom}}
\newcommand*{\CSrep}[1]{[#1]_{{\equiv}_{\CSdom}}}
\newcommand*{\CSleq}{\mathrel{\preceq}}%contrario implicazione
\newcommand*{\CSgeq}{\mathrel{\succeq}}%contrario implicazione
\newcommand*{\CSnleq}{\mathrel{\npreceq}}%contrario implicazione
\newcommand*{\CSngeq}{\mathrel{\nsucceq}}%contrario implicazione
\newcommand*{\CSmerge}{\mathbin{\wedge}}%merge constraints
\newcommand*{\CSjoin}{\mathbin{\vee}}%join constraints
\newcommand*{\CSbigmerge}{\bigwedge}
\newcommand*{\CSfalse}{\mathit{false}} %insieme di constraint
\newcommand*{\CStrue}{\mathit{true}} %insieme di constraint
\newcommand*{\CSc}[1][c]{#1}
\newcommand*{\instop}[2]{\ifempty{#1}{\downarrow}{{#1}\downarrow_{#2}}}
\newcommand*{\simpl}{\parensmathoper{\mathit{simplify}}}

\newcommand{\evalAExpr}[2]{\mathit{eval}_{\AExprDom}(#1,#2)}
\newcommand{\evalFAExpr}[2]{\widetilde{\mathit{eval}}_{\FAExprDom}(#1,#2)}
\newcommand{\evalBExpr}[2]{\mathit{eval}_{\BExprDom}(#1,#2)}
\newcommand{\evalFBExpr}[2]{\widetilde{\mathit{eval}}_{\FBExprDom}(#1,#2)}

\newcommand{\emptyPath}{\varepsilon}


\newcommand{\PathDom}{\mathit{Path}}
\newcommand{\paths}{\dot\Pi}
\newcommand{\pathsInterp}{\bar\Pi}
\newcommand{\botPaths}{\bot_{\PathDom}}
\newcommand{\leqPrefix}{\mathrel{\leq_{\mathit{prefix}}}}
\newcommand{\nleqPrefix}{\mathrel{\not{\leq}_{\mathit{prefix}}}}
\newcommand{\mcp}{\parensmathoper{\mathit{mcp}}}

\newcommand{\maxE}[2]{\mathit{max}\ifempty{#1}{}{\ifempty{#2}{(#1)}{(#1,#2)}}}
\newcommand{\minE}[2]{\mathit{min}\ifempty{#1}{}{\ifempty{#2}{(#1)}{(#1,#2)}}}


%% Program Transformation %%

\newcommand*{\betaPos}{\parensmathoper{\beta^{+}}}
\newcommand*{\betaNeg}{\parensmathoper{\beta^{-}}}
\newcommand*{\tauProg}{\parensmathoper{\tau}}
\newcommand*{\rerr}{\epsilon}
\newcommand*{\fperr}{\rerr}
\newcommand*{\fv}{\parensmathoper{\mathit{fv}}}


\newcommand*{\alphaeta}[2]{\alpha\ifempty{#1}{}{(#1,#2)}}
\newcommand*{\fpop}{\widetilde{\rop}}

%

\pagestyle{empty}
\begin{document}

\title{A Mixed Real and Floating-Point Solver}

\author{Rocco Salvia\inst{1} \and Laura Titolo\inst{2} \and  Marco A. Feli\'{u}\inst{2} \and Mariano M. Moscato\inst{2} \and C\'{e}sar A. Mu\~{n}oz\inst{3} \and Zvonimir Rakamari\'c\inst{1}}

\authorrunning{R. Salvia et al.}

\institute{
University of Utah\thanks{Partially supported by NSF awards CCF 1346756 and CCF 1704715.},\\
\email{\{rocco,zvonimir\}@cs.utah.edu}
\and
National Institute of Aerospace\thanks{Research by the first four authors was supported by the
National Aeronautics and Space Administration under NASA/NIA Cooperative Agreement NNL09AA00A.},\\
\email{\{laura.titolo,marco.feliu,mariano.moscato\}@nianet.org}%
\and
NASA Langley Research Center,\\
\email{cesar.a.munoz@nasa.gov}
}

\maketitle

\begin{abstract}
Reasoning about mixed real and floating-point constraints is essential
for developing accurate analysis tools for floating-point
programs. This paper presents FPRoCK, a prototype tool for solving
mixed real and floating-point formulas. FPRoCK transforms a mixed
formula into an equisatisfiable one over the reals. This
formula is then solved using an off-the-shelf SMT solver.
FPRoCK is also integrated with the PRECiSA static analyzer,
which computes a sound estimation of the round-off error of a
floating-point program. It is used to detect infeasible
computational paths, thereby improving the accuracy of PRECiSA.
\end{abstract}

\section{Introduction}
\label{sec:intro}
% 
Floating-point numbers are frequently used as an approximation of real numbers in computer programs.
%
A round-off error originates from the difference between a real number and its floating-point representation, and accumulates throughout a computation. The resulting error may affect both the computed value of arithmetic expressions as well as the control flow of the program.
% 
To reason about floating-point computations with possibly diverging control flows, it is essential to solve mixed real and floating-point arithmetic constraints.
%
This is known to be a difficult problem. In fact, constraints that are unsatisfiable over the reals may hold over the floats and vice-versa. In addition, combining the theories is not trivial since floating-point and real arithmetic do not enjoy the same properties.

Modern \emph{Satisfiability Modulo Theories} (SMT) solvers, such as \Mathsat~\cite{CimattiGSS13} and \Zt~\cite{MouraB08}, encode floating-point numbers with bit-vectors.
%
This technique is usually inefficient due to the size of the binary representation of floating-point numbers.
%
For this reason, several abstraction techniques have been proposed to approximate floating-point formulas and to solve them in the theory of real numbers.
% 
Approaches based on the \emph{counterexample-guided abstraction refinement} (CEGAR) framework \cite{BrilloutKW09,RamachandranW16,ZeljicBWR18} simplify a floating-point formula and solve it in a proxy theory that is more efficient than the original one.
%
If a model is found for the simplified formula, a check on whether this is also a model for the original formula is performed. If it is, the model is returned, otherwise, the proxy theory is refined.
%
\Realizer{}~\cite{LeeserMRW14} is a framework built on the top of \Zt{} to solve floating-point formulas by translating them into equivalent ones in real arithmetic.
%
\Molly{}~\cite{RamachandranW16} implements a CEGAR loop where floating-point constraints are lifted in the proxy theory of mixed real and floating-point arithmetics.
%
To achieve this, it uses an extension of \Realizer{} that supports mixed real and floating-point constraints.
%
However, this extension is embedded in \Molly{} and cannot be used as a standalone tool. 
% 
The \Colibri~\cite{MarreBC18} solver handles the combination of real and floating-point constraints by using disjoint floating-point intervals and difference constraints.
% 
Unfortunately, the publicly available version of \Colibri{} does not support all the rounding modalities and the negation of Boolean formulas.
%
JConstraints~\cite{jconstraints} is a library for constraint solving that includes support for
floating-points by encoding them into reals.

This paper presents a prototype solver for mixed real and floating-point constraints called \smttool{}.\footnote{The \smttool{} distribution is available at \url{https://github.com/nasa/FPRoCK}.}
% 
It extends the transformation defined in \Realizer{}~\cite{LeeserMRW14} to mixed real/floating-point constraints.
%
Given a mixed real-float formula, \smttool{} generates an equisatisfiable real arithmetic formula that can be solved by an external SMT solver.
%
In contrast to \Realizer, \smttool{} supports mixed-precision floating-point expressions and different ranges for the input variables.
%
\smttool{} is also employed to improve the accuracy of the static analyzer \tool~\cite{TitoloFMM18}.
% 
In particular, it identifies spurious execution traces whose path conditions are unsatisfiable, which allows {\tool{}} to discard them.


\section{Solving Mixed Real/Floating-Point Formulas}
\label{sec:smt}
% !TEX root = main.tex

A \emph{floating-point number}~\cite{IEEE754floating}, or simply a \emph{float}, can be represented by a tuple $(s, m, \mathit{exp})$ where $s$ is a sign bit, $m$ is an integer called the \emph{significand} (or \emph{mantissa}), and $\mathit{exp}$ is an integer \emph{exponent}.
%
A float $(s, m, \mathit{exp})$ encodes the real number $(-1)^{s} \cdot m \cdot 2^\mathit{exp}$.
%
Henceforth, $\Fp$ represents the set of floating-point numbers.
% 
Let $\fpv{}$ be a floating-point number that represents a real number
$\rv{}$. The difference $|\fpv{} - \rv{}|$ is called the \emph{round-off error} (or \emph{rounding error}) of $\fpv{}$ \wrt{} $\rv{}$.
%
Each floating-point number has a format $f$ that specifies its dimensions and precision, such as single or double.
%
The expression $\RtoF[f]{\rv{}}$ denotes the floating-point number in
format $f$ \emph{closest} to $\rv{}$ assuming a given rounding mode.

Let $\Var$ and $\FVar$ be two disjoint sets of variables representing real and floating-point values \resp{}.
%
The set $\AExprDom$ of mixed arithmetic expressions is defined by the grammar
%
\begin{equation*}
\AExpr  ::= \rcnst  \mid x \mid \fpcnst \mid \fvar \mid
             \AExpr \rop  \AExpr \mid 
             \AExpr \fpop \AExpr \mid \RtoF[f]{A},
\end{equation*}
% 
where $\rcnst \in \R$, $x \in \Var$, $\rop \in \{+,-,*,/,|\cdot|\}$
(the set of basic real number arithmetic operators),
$\fpcnst \in \Fp$, $\fvar \in \FVar$, $\fpop \in \{\tilde{+}_f,\tilde{-}_f,\tilde{*}_f,\tilde{/}_f\}$
(the set of basic floating-point arithmetic operators) and
$f \in\{\mathit{single},\mathit{double}\}$ denotes the desired precision for the result.
%
The rounding operator $\RtoF[f]{}$ is naturally extended to arithmetic expressions.
%
According to the IEEE-754 standard~\cite{IEEE754floating}, each
floating-point operation is computed in exact real arithmetic
and then rounded to the nearest float, \ie{} $\AExpr \fpop_f \AExpr =
\RtoF[f]{\AExpr \rop \AExpr}$.
%
Since floats can be exactly represented as real numbers, an explicit transformation is not necessary.
%
The set of mixed real-float Boolean expressions $\BExprDom$  is defined by the grammar
\begin{equation*}
   \BExpr ::= \true \mid \false \mid \BExpr \wedge \BExpr
               \mid \BExpr \vee \BExpr \mid \neg \BExpr
               \mid \AExpr < \AExpr \mid \AExpr = \AExpr,
\end{equation*}
%
where $\AExpr\in\AExprDom$.

The input to \smttool{} is a formula $\tilde{\phi} \in \BExprDom$ that may contain both real and floating-point variables and arithmetic operators.
%
Each variable is associated with a type (real, single or
double precision floating-point) and range that can be
either bounded, \eg{}, $[1,10]$, or unbounded, e.g., $[-\infty, +\infty]$.
%
The precision of a mixed-precision floating-point arithmetic operation is automatically detected and set to the maximum precision of its arguments.
%
Given a mixed formula $\tilde{\phi} \in \BExprDom$, \smttool{} generates a formula $\phi$ over the reals such that $\tilde{\phi}$ and $\phi$ are equisatisfiable.
%
Floating-point expressions are transformed into equivalent real-valued expressions using the approach presented in \cite{LeeserMRW14}, while the real variables and operators are left unchanged.
%
It is possible to define $x \fop y$ as
% 
\begin{equation}
    \label{eq:fop}
    x \fop y = \left(\frac{\round{\frac{x \rop y}{2^\mathit{exp}}\cdot 2^p}}{2^p}\right) \cdot 2^\mathit{exp},
\end{equation}
% 
where $p$ is the precision of the format, $\mathit{exp} = \mathit{max}\{i\in\Z \mid 2^i \leq |x \rop y|\}$, and
$\round{}: \R \rightarrow \mathit{Int}$ is a function implementing the rounding modality~\cite{LeeserMRW14}.
% 
Therefore, given a floating-point formula $\tilde{\phi}$, an equisatisfiable formula without floating-point operators is obtained by replacing every occurrence of $x \fop y$ using Equation~\eqref{eq:fop}.
% 
This is equivalent to replacing the occurrences of $x \fop y$ with a new fresh real-valued variable $v$ and imposing $v = x \fop y$.
% 
From Equation~\eqref{eq:fop} it follows that $v \cdot 2^{p-\mathit{exp}} = \round{(x \rop y) \cdot 2^{p-\mathit{exp}}}$.
% 
Thus, the final formula $\phi$ is
% 
\begin{equation}
    \phi \dfn \tilde{\phi}[v/x \fop y] \wedge v \cdot 2^{p-\mathit{exp}} = \round{(x \rop y) \cdot 2^{p-\mathit{exp}}},
\end{equation}
% 
where $\tilde{\phi}[v/x \fop y]$ denotes the Boolean formula $\tilde{\phi}$ where all the occurrences of $x \fop y$ are replaced by $v$.
%
The precision $p$ is a constant that depends on the chosen floating-point format, while $\mathit{exp}$ is an integer representing the exponent of the binary representation of $x \fop y$.


To find an assignment for the exponent $\mathit{exp}$, \smttool{} performs in parallel a sequential and binary search over the dimension of $x \fop y$, as opposed to the simple sequential search implemented in \Realizer{}.
%
The implementation of the function $\round{}$ depends on the selected rounding mode and can be defined using floor and ceiling operators (see \cite{LeeserMRW14} for details).
%
Therefore, the transformed formula $\phi$ does not contain any floating-point operators, and hence it can be solved by any SMT solver that supports the fragment of real/integer arithmetics including floor and ceiling operators.
%
\smttool{} uses three off-the-shelf SMT solvers as back-end procedures to solve the transformed formula: \Mathsat~\cite{CimattiGSS13}, \Zt~\cite{MouraB08}, and \CVCf~\cite{Barrett11}.
Optionally, the constraint solver \Colibri~\cite{MarreBC18} is also
available for use within \smttool{}.
% 
\smttool{} provides the option to relax the restriction on the minimum exponent to handle subnormal floats.
% 
This solution is sound in the sense that it preserves the unsatisfiability of the original formula.
% 
However, if this option is used, it is possible that \smttool{} finds an assignment to a float that is not representable in the chosen precision, and therefore is not a solution for the original formula.
%
Furthermore, \smttool{} currently does not support special floating-point values such as \emph{NaN} and \emph{Infinity}.

\section{Integrating \smttool{} in \tool}
\label{sec:precisa}
\tool{}\footnote{The \tool{} distribution is available at \url{https://github.com/nasa/PRECiSA}.} (Program Round-off Error Certifier via Static Analysis)~\cite{TitoloFMM18} is a static analyzer based on abstract interpretation~\cite{CousotC77}.
%
\tool{} accepts as input a floating-point program and automatically
generates a sound over-approximation of the floating-point round-off
error and a proof certificate in the Prototype Verification System (PVS)~\cite{OwreRS92} ensuring its correctness.
% 
For every possible combination of real and floating-point execution paths, \tool{} computes a \emph{conditional error bound} of the form $\cerr{\eta}{\widetilde{\eta}}{r}{e}{}{}$, where
%
$\eta$ is a symbolic path condition over the reals,
$\widetilde{\eta}$ is a symbolic path condition over the floats,
and $r, e$ are symbolic arithmetic expressions over the reals.
%
Intuitively, $\cerr{\eta}{\widetilde{\eta}}{r}{e}{}{}$ indicates that if the conditions $\eta$ and $\widetilde{\eta}$ are satisfied, the output of the program using exact real number arithmetic is $r$ and the round-off error of the floating-point implementation is bounded by~$e$.
%

\tool{} initially computes round-off error estimations in symbolic form so that the analysis is modular.
%
Given the initial ranges for the input variables, \tool{} uses the Kodiak global optimizer~\cite{NarkawiczM13} to maximize the symbolic error expression $e$.
% 
Since the analysis collects information about real and floating-point execution paths, it is possible to consider the error of taking the incorrect branch compared to the ideal execution using real arithmetic.
This happens when the guard of a conditional statement contains a floating-point expression whose round-off error makes the actual Boolean value of the guard differ from the value that would
be obtained assuming real arithmetic.
%
When the floating-point computation diverges from the real one, it is said to be \emph{unstable}.

For example, consider the function $\textit{sign}(\tilde{x}) = \; \tagif \tilde{x} \geq 0 \tagthen 1 \tagelse -1$.
%
\tool{} computes a set of four different conditional error bounds: 
$\{
\cerr{\real{\tilde{x}} \geq 0}{\tilde{x} \geq 0}{r=1}{e=0}{}{},
\cerr{\real{\tilde{x}} < 0}{\tilde{x} < 0}{r=-1}{e=0}{}{},
\cerr{\real{\tilde{x}} \geq 0}{\tilde{x} < 0}{r=-1}{e=2}{}{},
\cerr{\real{\tilde{x}} < 0}{\tilde{x} \geq 0}{r=1}{e=2}{}{}
\}$.
%
The function $\real{}: \FVar \ra \Var$ associates with the floating-point variable $\tilde{x}$ a variable $x \in \Var$ representing the real value of $\tilde{x}$.
% 
The first two elements correspond to the cases where real and floating-point computational flows coincide. In these cases, the error is 0 since the output is an integer number with no rounding error.
%
The other two elements model the unstable paths. In these cases, the
error is $2$, which corresponds to the difference between the output of the two branches.
%
\tool{} may produce conditional error bounds with unsatisfiable
symbolic conditions (usually unstable), which correspond to execution paths that cannot take place.
%
The presence of these spurious elements
affects the accuracy of the computed error bound.
%
For instance, in the previous example, if $|\real{\tilde{x}} - \tilde{x}| \leq 0$ both unstable cases can be removed, and the overall error would be $0$ instead of $2$.
%

Real and floating-point conditions can be checked separately using SMT solvers that support real and/or floating-point arithmetic.
%
However, the inconsistency often follows from the combination of the real and floating-point conditions.
%
In fact, the floating-point expressions occurring in the conditions are implicitly related to their real arithmetic counterparts by their rounding error.
%
Therefore, besides checking the two conditions separately, it is necessary to check them in conjunction with a set of constraints relating each arithmetic expression $\widetilde{\mathit{expr}}$ occurring in the conditions with its real number counterpart $\FtoRA{\widetilde{\mathit{expr}}}$. 
% 
$\FtoRA{\widetilde{\mathit{expr}}}$ is defined by simply replacing in $\widetilde{\mathit{expr}}$ each floating-point operation with the corresponding real one and by applying $\real{}$ to floating-point variables.

\smttool{} is suitable for solving such constraints thanks to its ability to reason about mixed real and floating-point formulas.
%
Given a set $\iota$ of ranges for the input variables, for each conditional error bound $c = \cerr{\eta}{\widetilde{\eta}}{r}{e}{}{t}$ computed by \tool{}, the following formula $\psi$ modeling the information contained in the path conditions is checked using \smttool{}:
% 
\begin{equation}
    \begin{aligned}[t]
    \psi \dfn \eta \wedge \widetilde{\eta} \wedge
    \bigwedge\{|\widetilde{\mathit{expr}} - \FtoRA{\widetilde{\mathit{expr}}}| \leq \epsilon
    \; \mid \; & \widetilde{\mathit{expr}} \text{ occurs in } \widetilde{\eta},\\
      &
     \widetilde{\mathit{expr}}\not\in\FVar, \widetilde{\mathit{expr}}\not\in\Fp,
     \epsilon = \mathit{max}(e)|_{\iota}\}
     \end{aligned}
\end{equation}
%
The value $\mathit{max}(e)|_{\iota}$ is the round-off error of $\widetilde{\mathit{expr}}$ assuming the input ranges in $\iota$, and it is obtained by maximizing the symbolic error expression $e$ with the Kodiak global optimizer.
If $\psi$ is unsatisfiable, then $c$ is dropped from the solutions computed by \tool{}.
%
Otherwise, a counterexample is generated that may help to discover cases for which the computation is diverging or unsound.

Since \smttool{} currently supports only the basic arithmetic operators, while \tool{} supports a broader variety of operators including transcendental functions, a sound approximation is needed for converting \tool{} conditions into a valid input for \smttool{}.
%
The proposed approach replaces in $\psi$ each floating-point (\resp{} real) arithmetic expression with a fresh floating-point (\resp{} real) variable.
% 
This is sound but not complete, meaning it preserves just the unsatisfiability of the original formula.
% 
In other words, if $\psi[v_i/\widetilde{\mathit{expr}_i}]_{i=1}^n$ is unsatisfiable it follows that $\psi$ is unsatisfiable, but if a solution is found for $\psi[v_i/\widetilde{\mathit{expr}_i}]_{i=1}^n$ there is no guarantee that an assignment satisfying $\psi$ exists. 
%
This is enough for the purpose of eliminating spurious conditional bounds since it assures that no feasible condition gets eliminated.
%
In practice, it is accurate enough to detect spurious unstable paths.
% 
When a path condition is deemed unsatisfiable by \smttool{}, \tool{} states such unsatisfiability in the PVS formal certificate.
%
For simple path conditions, this property can be automatically checked by PVS. Unfortunately, there are cases where human intervention is required to verify this part of the certificates.

\smartref{tbl:bm} compares the original version of \tool{} with the enhanced version that uses \smttool{} to detect the unsatisfiable conditions, along with the analysis tool Rosa~\cite{DarulovaK14} which also computes an over-approximation of the round-off error of a program.
%
All the benchmarks are obtained by applying the transformation defined in \cite{TitoloMFM18} to code fragments from avionics software and the FPBench library~\cite{DamoucheMPQST16}.
%
A transformed program is guaranteed to return either the result of the original floating-point program, when it can be assured that both its real and floating-point flows agree, or a warning when these flows may diverge.
% 
The results show that \smttool{} helps \tool{} improving the computed round-off error in 8 out of 11 benchmarks total.
% 
\smttool{} runs all search encoding (linear, binary) plus solver (MathSAT5, CVC4, Z3) combinations in parallel. It waits for all solvers to finish and performs a check on the consistency of the solutions.
% 
\begin{table}[tp]
\caption{Experimental results showing absolute round-off error bounds and execution time in seconds (best results in bold).}
   \begin{center}
    {\footnotesize
    \begin{tabular}{|l|r|r|r|r|r|r|}
    \hline
    \multicolumn{1}{|c|}{\multirow{2}{*}{Benchmark}} &
      \multicolumn{2}{c|}{\tool} &
      \multicolumn{2}{c|}{\tool+\smttool} &
      \multicolumn{2}{c|}{Rosa}\\
    \cline{2-7}
    & \multicolumn{1}{c|}{Error} & \multicolumn{1}{c|}{Time(s)} & \multicolumn{1}{c|}{Error} & \multicolumn{1}{c|}{Time(s)}  & \multicolumn{1}{c|}{Error} & \multicolumn{1}{c|}{Time(s)}  \\
    \hline
    \hline
    cubicSpline        & 2.70E+01  & \textbf{0.07}  & 2.70E+01 & 97.8    & \textbf{2.50E-01}  & 24.1  \\        
    eps\_line          & 2.00E+00  & \textbf{0.02}  & \textbf{1.00E+00} & 48.8   & 2.00E+00  & 15.5  \\        
    jetApprox          & 1.51E+01  & \textbf{12.79}  & 8.11E+00 & 263.3  & \textbf{4.97E+00}  & 924.8 \\         
    linearFit          & 1.08E+00  & \textbf{0.06}  & 5.42E-01 & 259.7  & \textbf{3.19E-01}  & 12.4  \\        
    los                & 2.00E+00  & \textbf{0.02}  & \textbf{1.00E+00} & 46.2   & not supported       & n/a     \\  
    quadraticFit       & 3.68E+00  & \textbf{0.90}  & 3.68E+00 & 259.8  & \textbf{1.27E-01}  & 82.4  \\        
    sign               & 2.00E+00  & \textbf{0.02}  & \textbf{1.00E+00} & 32.1    & 2.00E+00  & 4.7   \\       
    simpleInterpolator & 2.25E+02  & \textbf{0.03}  & 1.16E+02 & 93.8   & \textbf{3.33E+01}  & 6.3   \\       
    smartRoot          & \textbf{1.75E+00}  & \textbf{0.32}  & \textbf{1.75E+00} & 0.6    & not supported      & n/a     \\  
    styblinski         & 9.35E+01  & \textbf{1.06}  & 6.66E+01 & 260.1  & \textbf{6.55E+00}  & 77.0  \\        
    tau                & 8.40E+06  & \textbf{0.03}  & \textbf{8.00E+06} & 101.8  & 8.40E+06  & 20.7  \\     
    \hline
  \end{tabular}}
\end{center}
  \label{tbl:bm}
\end{table}%

\section{Conclusions}
\label{sec:conclusion}
% 
This paper presents \smttool{}, a prototype tool for solving mixed real and floating-point formulas.
%
\smttool{} extends the technique used in \Realizer{} by adding support
for such mixed formulas.
%
\smttool{} is integrated into PRECiSA to improve its precision.
% 
Similarly, it could be integrated into other static analyzers, such as \FPTaylor~\cite{SolovyevJRG15}.
%
% 
The current version of \smttool{} has some limitations in terms of expressivity and efficiency.
%
Support for a vast range of operators, including transcendental functions, is contingent on the expressive power of the underlying SMT solvers.
%
The performance of \smttool{} can be improved by returning a solution as soon as the first solver finalizes its search.
% 
However, finding an assignment for the exponent of each floating-point variable is still the major bottleneck of the analysis.
%
The use of a branch-and-bound search to divide the state-space may help to mitigate this problem.

\bibliographystyle{splncs04}
\bibliography{biblio}

\end{document}

\grid
