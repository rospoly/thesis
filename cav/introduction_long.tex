\section{Introduction}
\label{sec:introduction}

There are two common sources of randomness in a numerical computation (a
straight-line program). First, the computation might be using inherently noisy data, for example from analog sensors in cyber-physical systems such as robots, autonomous vehicles, and
drones.  A prime example is data from GPS sensors, whose error distribution can be described very precisely~\cite{bornholt2014uncertain} and which we study in some detail in \cref{sec:overview}.
Second, the computation itself might sample from random number
generators. Such probabilistic numerical routines, known as Monte-Carlo
methods, are used in a wide variety of tasks, such as integration~\cite{press1990recursive,lepage1980vegas},
optimization~\cite{press1988numerical}, finance~\cite{glasserman2013monte}, fluid
dynamics~\cite{landau2014guide}, and computer graphics~\cite{kajiya1986rendering}.  We call
 numerical computations whose input values are sampled from
some probability distributions \emph{probabilistic computations}.  


Probabilistic computations are typically implemented using
floating-point arithmetic, which leads to roundoff errors being introduced in
the computation. To strike the right balance between the performance
and energy consumption versus the quality of the computed result, expert
programmers rely on either a manual or automated floating-point error analysis
to guide their design decisions. However, the current state-of-the-art
approaches in this space have primary focused on \emph{worst-case} roundoff error
analysis of \emph{deterministic} computations.  So what can we say about
floating-point roundoff errors in a probabilistic context? Is it possible to
probabilistically quantify them by computing confidence intervals?  Can we, for
example, say with 99\% confidence that the roundoff error of the computed
result is smaller than some chosen constant?  What is the distribution of
outputs when roundoff errors are taken into account? In this paper, we explore these and similar questions.  To answer them, we propose a rigorous -- that is to say \emph{sound} -- approach to quantifying
roundoff errors in probabilistic computations. Based on this
approach, we develop an automatic tool that efficiently computes an overapproximate
probabilistic profile of roundoff errors.

As an example, consider the floating-point arithmetic expression $(X+Y)\div Y$, where $X$ and $Y$ are random inputs described formally as independent random variables (usually defined over some intervals). In \cref{sec:errordist}, we first show how the computation in \emph{finite-precision} of a single arithmetic operation such as $X+Y$ can be modelled as $(X+Y)(1+\varepsilon)$, where $\varepsilon$ is also a random variable. We then show how this random variable can be computed from first principles and why it makes sense to view $(X+Y)$ and  $(1+\varepsilon)$ as independent expressions, which in turn allows us to easily compute the distribution of $(X+Y)(1+\varepsilon)$.
The distribution of $\varepsilon$ depends on that of $X+Y$, and we therefore need to evaluate arithmetic operations between random variables. When the operands are independent -- as in $X+Y$ -- this is standard~\cite{springer1979algebra}, but when the operands are dependent -- as in the case of the division in $(X+Y)\div Y$ -- this is a hard problem. To solve it, we adopt and improve a technique for soundly bounding these distributions described in~\cite{bouissou2012generalization}.  Our improvement comes from the use of an SMT solver to reason about the dependency between $(X+Y)$ and $Y$ and remove regions of the state-space with zero probability. We describe this in \cref{sec:model}.

We can thus soundly bound the output distribution of any probabilistic computation, such as $(X+Y)\div Y$, performed in finite-precision floating-point arithmetic. This gives us the ability to perform \emph{probabilistic range analysis} and prove rigorous assertions like: 99\% of the outputs of a floating-point computation are smaller than a given constant bound.
In order to perform \emph{probabilistic roundoff error analysis} we develop \emph{symbolic affine arithmetic} in \cref{sec:symbolicaffine}. This technique is combined with probabilistic range analysis to compute \emph{conditional roundoff errors}. Specifically, we over-approximate the maximal error conditioned on the output landing in the 99\% range computed by the probabilistic range analysis, \ie conditioned on the computations not returning an outlier.  This allows us to quantify very precisely the idea of the maximal error of a `typical' computation.

Here we need to make a comment about how to understand the notion of \emph{sound} or \emph{rigorous} error bounds in a probabilistic context.  The usual meaning of these words is anchored in a non-deterministic model of computation where inputs are known to belong to some intervals and soundly bounding the error amounts to bounding the error of \emph{all possible computations}.
%
If the inputs are drawn from a distribution, this understanding of soundness corresponds to bounding errors with 100\% confidence, and thereby ignoring the probabilistic information completely.
%
This is why the state-of-the-art refers to this non-deterministic model as \emph{worst-case} analysis, because it does not account for the distribution of the inputs.
%
On the other hand, in a probabilistic model of computation,  we will say that we are giving a sound (or rigorous) probabilistic guarantee if there is 100\% certainty about the probabilistic guarantee.   For example when saying \emph{99\% of outputs will land in the interval $[a,b]$} we are making a sound probabilistic statement based on some analytical description of the output distribution, and there is therefore no uncertainty about this statement.  An example of \emph{unsound} probabilistic statement is given by making probabilistic statements based on Monte-Carlo sampling. When saying \emph{based on 1 billion runs, 99\% of outputs will land in the interval $[a,b]$} we are making an unsound probabilistic statement which may fail in another run of the experiment. Another example of unsound probabilistic statement would be \emph{with 95\% certainly, 99\% of outputs will land in the interval $[a,b]$}, as is done in \eg PAC learning \cite{valiant1984theory}.  By working with `p-boxes'~\cite{bouissou2012generalization} -- structures which upper- and lower-bound cumulative distribution functions (see \cref{sec:preliminaries} and \cref{sec:errordist}) -- we will be able to give probabilistic guarantees which are all \emph{sound}.

We implemented our model and algorithms in a tool called \Tool (for Probabilistic Analysis of Floating-point errors). We evaluated \Tool on the standard floating-point benchmark suite FPBench~\cite{fpbench}, and compared its range and error analysis with the worst-case roundoff error analyzer FPTaylor~\cite{2015_fm_sjrg,solovyev2018rigorous} and the probabilistic roundoff error analyzer PrAn~\cite{probdaisy}. We present the results in \cref{sec:evaluation},  and show that FPTaylor's worst-case analysis is often overly pessimistic in the probabilistic setting, while \Tool also generates tighter probabilistic error bounds than PrAn on all but one benchmark. 

We summarize our contributions as follows:
\begin{enumerate}[(i)]
\item We derive several closed-form expressions for the distribution of roundoff errors associated with a random variable.  We prove that roundoff errors are generally close to being uncorrelated with their input distribution.
\item Based on these results we propose a model of IEEE754 floating-point arithmetic for numerical expressions with probabilistic inputs.
\item We evaluate this model by developing a new algorithm for rigorously bounding the output range and roundoff error distributions of floating-point arithmetic expressions with probabilistic inputs.
\item We implement this model in the \Tool tool, and perform probabilistic range and roundoff error analysis on a standard benchmark suite. Our comparison
	with the current state-of-the-art shows the advantages of our approach in terms of computing tighter, and yet still rigorous,
	probabilistic bounds.
\end{enumerate}

