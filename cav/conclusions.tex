\section{Conclusions and Future Work}
\label{sec:conclusions}

In this paper, we presented our approach to computing rigorous (i.e., sound)
probabilistic roundoff error bounds for arithmetic floating-point computations
involving random variables. 
%
First, we showed how to explicitly compute the error distribution of
floating-point arithmetic operations directly from the distributions of the
input random variables.
%
Then, we demonstrated how the random variable and the corresponding error
distribution are close to being uncorrelated.
%
We leverage this to be able to rigorously compute operations between random
floating-point variables, and thus perform rigorous probabilistic range
analysis.
%
We showed how to do this even in the complex case of dependent operands using
our novel combination of p-boxes and SMT solving.
%
We use our probabilistic range analysis to compute conditional roundoff errors
where, as opposed to the worst-case approach, the (symbolic) error expression
gets maximized constrained to the output landing in a given confidence interval
of interest (e.g., 99\%). 
%
We implemented our approach in a prototype tool named \Tool, and we compared
\Tool on a popular benchmark suite with state-of-the-art tools for
probabilistic as well as worst-case error analysis.
%
Our results show that \Tool almost always outperforms the state-of-the-art
probabilistic analysis tool PrAn in terms of generating tighter, but still
rigorous, error bounds.
%
Moreover, we observe that worst-case roundoff errors can be very pessimistic in
some cases, and that \Tool can reduce such error bounds by several orders of
magnitude.



As future work, we plan to explore the potential use of our probabilistic range
analysis to perform probabilistic overflow detection.
%
Together with a warning about a potential overflow, a user would also
get from \Tool a rigorous bound on the probability of the overflow happening.
%
This would allow users to perform a more detailed risk analysis in order to decide
whether a mitigation effort is needed.
%
We envision a similar use case for the probabilistic division-by-zero
detection.
%
Finally, in our motivating example (see \cref{sec:overview}), we performed
manual probabilistic precision tuning to further drive the point that
probabilistic error analysis is needed in many settings.
%
In fact, \Tool can be used as a back-end roundoff error analyzer as a part of
an existing precision tuner (such as FPTuner~\cite{fptuner}).
%
In such a setup, the confidence interval becomes a new key parameter in the
precision tuning process, thereby allowing for programmers to explore a richer
space of trade-offs.
%
We plan to explore this line of work in the future.

