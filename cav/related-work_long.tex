\section{Related Work}
\label{sec:related-work}

Our work draws inspiration from the seminal probabilistic affine arithmetic
approach of Bouissou et al.~\cite{bouissou2012generalization}.
%
The fundamental object in this approach is the \emph{probabilistic affine
form}, which is an affine form whose error symbols are associated with p-boxes
(see \cref{subsec:prob}).
%
The authors propose an innovative technique to perform dependent operations
between random variables based on affine arithmetic. 
%
However, their approach can only detect affine dependencies between the
operands.
%
%On the other hand, in \Tool, we create proper augment the purely affine forms with SMT solver
%invocations.
%
On the other hand, in \Tool, we detect dependencies between operands using an SMT solver. 
%
Our approach allows us to handle not only polynomial dependencies, but also any non-linearities the solver supports.
%
%In case the operations are independent, \Tool uses the same approach described by the authors.
%
%
%This improvement allows us to handle not only polynomial dependencies, but also any non-linearity the solver supports.
%
Finally, their approach only computes output ranges, while we went a step
further and can compute roundoff error bounds as well, which is a non-trivial
extension.


The most similar approach to our work is the recent static probabilistic
roundoff error analysis called PrAn~\cite{probdaisy}; to the best of our
knowledge, this is also the only work apart from this paper that presents a
rigorous and general probabilistic roundoff error analysis.
%
PrAn builds on the probabilistic affine arithmetic of Bouissou et
al.~\cite{bouissou2012generalization}, and inherits the same limitations in
dealing with dependent operations we described above. 
%
%Like us, PrAn combines it with random inputs described as random variables.
%
Like us, their method hinges on a discretization scheme that builds
p-boxes for both the input and error distributions and propagates them through
the computation.
%
The question of how these p-boxes are chosen
is left open.  In contrast,  we take the input variables to be
random variables that the user specifies and show how the distribution of
each error terms can be computed directly and exactly from the random variable
generating it (see \cref{sec:errordist}).
%
%However, PrAn uses p-boxes indiscriminately and does not distinguish between
%dependent and independent operations.
%
%\Tool, on the other hand, distinguishes between dependent and independent
%operations --- it uses p-boxes only to compute arithmetic operations on
%dependent variables, while it computes the operations on independent variables
%using a more precise analytical approach.
%
%Furthermore, unlike PrAn, \Tool leverages an SMT solver to prune away
%infeasible combinations of p-boxes, which leads to improved over-approximations
%of arithmetic operations between dependent random variables.
%
Furthermore, unlike PrAn, \Tool leverages the noncorrelation between random variables and the corresponding error distribution we described in \cref{subsec:covar}.
%
The immediate consequence is that \Tool performs the rounding in \cref{eq:probabilistic} as an \emph{independent} operation, thus without any uncertainty.
%
%Moreover, unlike PrAn, \Tool uses the un-correlation between random variables and the corresponding error distributions (see~\cref{subsec:covar}) to reliably perform the rounding in~\cref{eq:probabilistic} as an \emph{independent} operation.
%
Putting these together leads to \Tool computing tighter probabilistic roundoff
error bounds than PrAn, as our experimental results show (see
\cref{sec:evaluation}).

%
The idea of using a probabilistic model of rounding errors to analyze \emph{deterministic} computations can be traced back to \cite{von1947numerical}.  Parker's so-called `Monte Carlo arithmetic' \cite{parker2000monte} is probably the most detailed description of this approach.  We, however,  consider \emph{probabilistic} computations,  and are therefore able to derive the error distribution from first principles.
For this reason, the famous critique of the probabilistic approach to roundoff errors \cite{kahan1996improbability} does not apply to this work, since it only considers deterministic computations.  When dealing with probabilistic computations however, everything -- including rounding errors -- has a probabilistic behavior and the probabilistic approach is therefore a  necessity.  
%Our probabilistic approach is formally not new, it can be traced back to \cite{von1947numerical} and is similar to the Monte Carlo arithmetic of \cite{parker2000monte},  but our derivation of the error distribution and our theoretical justification of the model is.
%\zvon{Fred, please check the previous sentences.}
%The famous critique of the probabilistic approach to roundoff errors by Kahan~\cite{kahan1996improbability} does not apply to our work because we consider programs with probabilistic inputs. Everything about the behaviour of such a programs is probabilistic, including roundoff errors, and a probabilistic approach is just a necessity.  
%\cite{kahan1996improbability} on the other hand, critiques the probabilistic modelling of rounding errors in \emph{deterministic programs}.  
%We show in \cref{subsec:LPerror_dist,} and \cref{subsec:HPerror_dist} that rounding error distributions  do not need to be \emph{modelled}, rather they can be \emph{evaluated} exactly. This contrasts with the methodological issues raised by~\cite{kahan1996improbability}.



More recently, probabilistic roundoff error models have been investigated by Higham and Mary~\cite{higham2019new} as well as Ipsen and Zhou~\cite{ipsen2019probabilistic}. Interestingly, because these authors are interested in large-dimensional problems, neither approach needs to explicitly specify the distribution of errors.  However, this means that these approaches are only applicable to large-dimensional problems, and are completely unsuited for the domains (e.g., control systems, filters, cyber-physical systems)  captured by the FPBench benchmark suite. Finally, unlike us, neither approach is sound since they rely on concentration of measure inequalities.

%Instead, Higham and Mary require that each sample from $dist$ be independent and that $\Exp{\delta}=0$, whilst Ipsen and Zhou require only $\Exp{\delta}=0$. By using concentration of measure inequalities the authors then obtain probabilistic bounds which are independent of any particular choice of distribution. 
%These bounds, however, are only applied to a small class of problems (inner product, matrix multiplication, and matrix factorisation algorithms) with very large inputs. We, on the other hand, explicitly construct the error distribution $dist$ for a given input distribution, and use this construction to give a probabilistic model of IEEE floating-point arithmetic that can be applied to small and medium-sized programs in a compositional way. 
%


Unlike probabilistic roundoff error analysis, worst-case analysis of roundoff errors has been an active research area
with numerous published approaches\\
\cite{gappa,gappapp,smartfloat,fluctuat,rangelab,precisa,darulova2018daisy,2015_fm_sjrg,solovyev2018rigorous,rosa,magron2017certified,pldi16-bit-level-fp-verification,zhendong2015,satire}. As expected, none of them can deal with probabilistic inputs.  Affine arithmetic~\cite{affineoriginal} is a well-known technique that extends interval arithmetic~\cite{intervalarithmetic} with information about linear correlations between operands~\cite{affinebook}.
%
Rigorous affine arithmetic~\cite{rigorousaffine} takes into consideration numerical errors stemming from the use of floating-point arithmetic when computing with affine forms, and it has been implemented in Rosa~\cite{rosa} to perform rigorous worst-case roundoff error analysis.
%
Our symbolic affine arithmetic (see \cref{sec:symbolicaffine}) used in \Tool evolved from rigorous affine arithmetic by keeping the coefficients of the noise terms symbolic, which often leads to improved precision.
%
These symbolic terms are very similar to the first-order Taylor approximations
of the roundoff error expressions used in
FPTaylor~\cite{2015_fm_sjrg,solovyev2018rigorous}; our experimental results
also support this since the bounds generated by \Tool for the 100\% confidence
interval are almost always equal to the worst-case bounds computed by FPTaylor.

