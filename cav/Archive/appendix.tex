
\appendix

\section{Proofs}
We define $\ceil{x}\defeq \sup\{y\in\R\mid \round(y)=\round(x)\}$ and $\floor{x}\defeq \inf\{z\in\R\mid \round(z)=\round(x)\}$. Recall also that 
\[
\fintvl\stackrel{\triangle}{=}\left\{y\in\R\mid \round(y)=\round(z)\right\}.
\]

\subsection{Proofs for \cref{subsec:LPerror_dist}.}

\begin{myproof}{\cref{thm:LP_errorDensity}}
For $t\in\left]-1,1\right[$, the cumulative distribution function of the measure associated with the random variable $\erel(X)$ is given by:
\begin{align*}
c(t)\stackrel{\triangle}{=}~\Pro{\erel(X)\leq t\uro}=~\Pro{~\bigvee_{z\in\F}\left(\frac{X-z}{X}\leq t\uro\wedge X\in \fintvl[z]\right)}
\end{align*}
As explained in \cref{subsec:LPerror_dist} we exclude the floating-point representable numbers $\{-\infty, 0, \infty\}$ which correspond to the discrete components $\Pro{X\ssvin\fintvl[0]}\delta_1 + \Pro{X\ssvin\fintvl[-\infty]}\delta_{\infty}\ \hspace{-3pt}+ \Pro{X\ssvin\fintvl[\infty]}\delta_{-\infty}$ of the distribution $dist$.

Using the $\sigma$-additivity of measures (see \cref{subsec:prob}) we get the following density for $dist_c$:
\newcommand{\ssin}{\hspace{-2pt}\in\hspace{-2pt}}
\begin{align*}
d(t)=&\dt\sum_{z\in\F\setminus\{-\infty,0,\infty\}}\hspace{-8pt}\Pro{\frac{X-z}{X}\leq t\uro\wedge X\in \fintvl[z]}\\
=&\hspace{-6pt} \sum_{z\in\F^+_{0,-\infty}}\hspace{-8pt}\dt\Pro{\frac{z}{1-t\uro}\hspace{-2pt}\geq\hspace{-2pt} X\wedge X\ssin\fintvl[z]}+\hspace{-4pt}
\sum_{z\in\F^-_{0,\infty}}\hspace{-4pt}\dt\Pro{\frac{z}{1-t\uro}\hspace{-2pt}\leq\hspace{-2pt} X \wedge X\ssin \fintvl[z]}
\end{align*}
where $\F^+_{0,\infty}$ (resp. $\F^-_{0,-\infty}$) denotes the strictly positive (resp. negative) finite floating-point representable numbers.
Since $X$ is described by a probability density function $f:\R\to\R$, we get:
\begin{align}
d(t)
=&\hspace{-6pt}\sum_{z\in\F^+_{0,-\infty}}\hspace{-8pt}\dt\one_{\fintvl[z]}\hspace{-2pt} \left(\frac{z}{1-t\uro}\right)\hspace{-2pt} \int^{\frac{z}{1-t\uro}}_{\floor{z}}\hspace{-4pt}f(s)\hspace{1pt}ds\nonumber+
\hspace{-4pt}\sum_{z\in\F^-_{0,\infty}}\hspace{-6pt} \dt\one_{\fintvl[z]}\hspace{-2pt}\left(\frac{z}{1-t\uro}\right)\hspace{-2pt} \int^{\ceil{z}}_{\frac{z}{1-t\uro}}\hspace{-5pt} f(s)\hspace{1pt}ds\nonumber 
\\
=&\sum_{z\in\F\setminus\{-\infty,0,\infty\}}\hspace{-5pt}\one_{\fintvl[z]}\left(\frac{z}{1-t\uro}\right) f\left(\frac{z}{1-t\uro}\right) \frac{\uro\absv{z}}{(1-t\uro)^2}\nonumber
\end{align}
\end{myproof}

\subsection*{Proofs for \cref{subsec:HPerror_dist}.}

We now give some useful results about floating-point numbers.
The following lemma collects explicit representations of $\ceil{z}$ and $\floor{z}$ which will be useful in what follows, the proof is by direct computation.
\begin{lemma}\label{lem:floorceil}
For $z=z(0,e,k)\in \F$ the values of $\ceil{z}$ and $\floor{z}$ are given by: 
\begin{align*}
\floor{z}& =
\begin{cases}
2^{e-1} & \text{if }e=\emin, k=0\\
2^{e-1}\left(1+\frac{2^{p+1}-1}{2^{p+1}}\right) & \text{if }e>\emin, k=0\\
2^e\left(1+\frac{2k-1}{2^{p+1}}\right) & \text{otherwise}
\end{cases} 
& \qquad & \\
\ceil{z}&=\begin{cases}
z&\text{if }e=n, k=2^p-1\\
2^e\left(1+\frac{2k+1}{2^{p+1}}\right) & \text{otherwise}
\end{cases}
\end{align*}
For $z=z(1,e,k)$ we use the identities $\floor{-z}=-\ceil{z}$ and $\ceil{-z}=-\floor{z}$.
\end{lemma}

Using this lemma we can prove a result relating the length $\tau(z)\defeq\ceil{z}-\floor{z}$ of the interval corresponding to a representable number $z\in\F$ with its absolute value $\absv{z}$. Note that $\ceil{z}-\floor{z}$ is always a positive quantity, hence the relation with $\absv{z}$. The following result will be used heavily in the proof of \cref{thm:HP_errorDensity} . Again, the proof can be obtained by direct computation.

\begin{lemma}\label{lem:Ccoeff}
For any $z(s,e,k)\in\F, z\neq 0$ it is the case that 
\[
u\absv{z} =  C(e,k)\left(\ceil{z}-\floor{z}\right)
\]
where the coefficients $C(e,k)$ are given by:
\begin{align*}
&C(\emin,0) = \frac{2^{p+1}+1}{2^p(2^{p+1}-1)} & & C(e,0)=\frac{2}{3},\quad e>\emin\\
&C(\emax,2^p-1)=\frac{3(2^{p+1}-1)}{2^{p+1}} & & C(e,k)=\frac{2^p+k}{2^{p+1}},\quad\text{otherwise}.
\end{align*}
\end{lemma}
\noindent Finally, we will need the following technical lemma which can also be shown by direct computation.
\begin{lemma}\label{lem:trange}
If $z=z(s,e,k)$ with $e\neq \emin,\emax$ then 
\[
\frac{z}{1-t\uro}\in\fintvl \text{ iff } 
\begin{cases}
\frac{-2^{p+1}}{2^{p+2}-1}\leq t\leq \frac{2^{p+1}}{2^{p+1}+1} & k=0\\
\frac{-2^{p+1}}{2^{p+1}+2k-1}\leq t\leq \frac{2^{p+1}}{2^{p+1}+2k+1}&\text{otherwise}
\end{cases}
\]
\end{lemma}
Note in particular that $\frac{z}{1-t\uro}\in\fintvl$ whenever $\absv{t}\leq \frac{1}{2}$. We are now ready to prove the theorem of \cref{subsec:HPerror_dist}.


\begin{myproof}{\cref{thm:HP_errorDensity}}
We start by plugging the coefficients of Prop. \ref{lem:Ccoeff} into \cref{eq:LPerrorDensity}:
\begin{align*}
d(t)=\sum_{z\in\F\setminus\{-\infty,0,\infty\}}\one_{\fintvl[z]}\left(\frac{z}{1-t\uro}\right) f\left(\frac{z}{1-t\uro}\right) \frac{C(e,k) (\ceil{z}-\floor{z})}{(1-t\uro)^2}.
\end{align*}
We start by considering the case where $\absv{t}\leq \frac{1}{2}$. As shown in \cref{lem:trange} if $\absv{t}\leq \frac{1}{2}$ then $\one_{\fintvl[z]}\left(\frac{z}{1-t\uro}\right) = 1$, and $d(t)$ thus simplifies to
\begin{align*}
d(t)=\sum_{z\in\F\setminus\{-\infty,0,\infty\}} f\left(\frac{z}{1-t\uro}\right) \frac{C(e,k) (\ceil{z}-\floor{z})}{(1-t\uro)^2}.
\end{align*}
We now restrict ourselves to floating-point representable whose exponents are not extremal, \ie $z(s,e,k)$ with $k\notin\{\emin,\emax\}$, and put the usually minuscule contribution made by these numbers in the error term $R$, \ie we will take
 \[
 \Pro{\round(X)=z(s,\emin, k)} +  \Pro{\round(X)=z(s,\emax, k)} 
 \]
to be part of the error term $R$. We do this chiefly for mathematical expediency (to avoid the special behaviour of \cref{lem:Ccoeff}).
We are now left with the sum 
\begin{align}
d(t) =& \frac{1}{(1-t\uro)^2} \sum_{s, e=\emin+1}^{\emax-1} \left(\frac{2}{3}f\left(\frac{z(s,e,0)}{1-t\uro}\right)(\ceil{z(s,e,0)}-\floor{z(s,e,0)})\right. \nonumber\\
&
+\left. \sum_{k=1} \frac{2^p+k}{2^{p+1}} f\left(\frac{z(s,e,k)}{1-t\uro}\right)(\ceil{z(s,e,k)}-\floor{z(s,e,k)}) \right)\nonumber
\\
\stackrel{(1)}{=} & \frac{1}{(1-t\uro)^2} \sum_{s, e=\emin+1}^{\emax-1} \left(\frac{1}{2}f\left(\frac{z(s,e,0)}{1-t\uro}\right)(\ceil{z(s,e,0}-z(1-\uro))\right.\nonumber\\
&
+\left. \sum_{k=1}^{2^{p-1}} \frac{z(s,e,k)}{2^{e+1}} f\left(\frac{z(s,e,k)}{1-t\uro}\right)(\ceil{z(s,e,k)}-\floor{z(s,e,k)}) \right)
\nonumber\\
\stackrel{(2)}{=} & \frac{1}{1-t\uro} \sum_{s, e=\emin+1}^{\emax-1}\frac{1}{2^{e+1}} \left(\frac{z(s,e,0)}{1-t\uro}f\left(\frac{z(s,e,0)}{1-t\uro}\right)(\ceil{z(s,e,0}-z(1-\uro))\right.\nonumber\\
&
+\left. \sum_{k=1}^{2^{p-1}} \frac{z(s,e,k)}{1-t\uro} f\left(\frac{z(s,e,k)}{1-t\uro}\right)(\ceil{z(s,e,k)}-\floor{z(s,e,k)}) \right)
\nonumber\\
\stackrel{(3)}{=} &  \frac{1}{1-t\uro} \sum_{s, e=\emin+1}^{\emax-1}\frac{1}{2^{e+1}} \left(\int_{(-1)^s2^e(1-u)}^{(-1)^s2^e(2-u)} \absv{x} f(x) ~dx \right) + S(e,s,t)\label{eq:case_one_half}
\end{align}
where $(1)$ follows by computing the distance $\ceil{z(s,e,0)}-\floor{z(s,e,0)}$ with \cref{lem:floorceil} and by substituting $k$ for its value in terms of $z$ via $z=2^e(1+\frac{k}{2^p})$, and $(2)$ follows from writing $\frac{1}{2}$ as $\frac{z(s,e,0)}{2^{e+1}}$. The last step follows by noticing that the sum in step (2) approximates the integral in (3) (where the absolute value $\absv{x}$ takes care of the fact that the bounds of the integral will be the 'wrong way round' when $s=1$).  Indeed, when $t=0$ it corresponds to the  so-called \emph{mid-point rule} of numerical integration. We can explicitly compute the error $S(e,s,t)$ incurred from this approximation using standard technique from numerical analysis. Specifically,  for a fixed $z(s,e,k)$ we write the function $xf(x)$ as a Taylor expansion around $\frac{z}{1-tu}f\left(\frac{z}{1-tu}\right)$ with a first-order remainder in Lagrange form to get
\begin{align}
xf(x) = \frac{z}{1-tu}f\left(\frac{z}{1-tu}\right) + \left(x- \frac{z}{1-tu}\right)\left(f'(\xi_x)\xi_x + f(\xi_x)\right)\label{eq:Taylor}
\end{align}
for some $\xi_x$ between $x$ and  $\frac{z}{1-tu}$.  Using this representation we get the error term
\begin{align*}
S(z, t) \defeq & \int_{z_0}^{\ceil{z}} f\left(\frac{z(s,e,k)}{1-t\uro}\right)(\ceil{z(s,e,k)}-\floor{z(s,e,k)})  - \int_{z_0}^{\ceil{z}} xf(x)~dx
\\
& = \left(f'(\xi_k)\xi_k + f(\xi_k)\right) \frac{-zt\uro}{1-t\uro}\left(\ceil{z}-z_0\right)
\end{align*}
where $z_{0}=z(1-\uro)$ when the significand of $z$ is 0 and $z_0=\floor{z}$ otherwise. The last step uses the Mean Value Theorem extract a unique $\xi_k\in[z_0,\ceil{z}]$.. Note that when $t=0$ the error term above disappears. This is completely expected since $t=0$ corresponds to the mid-point rule which is a second-order numerical integration technique (it has no first-order error). We should therefore expand \cref{eq:Taylor} to the second order in the case where $t=0$. however, since we will integrate over every $t$, what happens at $t=0$ can be ignored since the Lebesgue measure is continuous.  Now summing over every significand and using the Mean Value Theorem once more we get and cumulative error
\begin{align*}
S(e,s,t)& \defeq \sum_{k=0}^{2^p-1}S(z(s,e,k), t) \\
&= \left(f'(\xi_{e,s})\xi_{e,s} + f(\xi_e)\right)\sum_{k=0}^{2^p-1}\frac{-z(s,e,k)t\uro}{1-t\uro}\left(\ceil{z(s,e,k)}-z_0\right)
\end{align*}
for some $\xi_{e,s}$ in $[2^e(1-\uro), 2^e(2-\uro)]$ when $s=0$ and  $[-2^e(2-\uro), -2^e(1-\uro)]$ when $s=1$,  and for $z_0$ defined as above. We can bound $\absv{S(e,s,t)}$ as follows. We first compute
\begin{align*}
\absv{\frac{-zt\uro}{1-t\uro}\left(\ceil{z}-z_0\right))}&=\absv{\frac{t}{1-t\uro}} \absv{z}\uro\left(z-z_0\right)\\
&\stackrel{(1)}{=}\absv{\frac{t}{1-t\uro}} \frac{2^p+k}{2^{p+1}}\left(z-z_0\right)^2\\
&\stackrel{(2)}{=}\absv{\frac{t}{1-t\uro}} \frac{2^p+k}{2^{p+1}}\frac{2^{2e}}{2^{2p}}
\end{align*}
where (1) follows by \cref{lem:Ccoeff} and (2) follows from the fact that all the intervals $z-z_0$ have the same size given by $\nicefrac{2^e}{2^p}$. We can now bound $S(e,s,t)$ by
\begin{align}
\absv{S(e,s,t)}&= \absv{f'(\xi_{e,s})\xi_{e,s} + f(\xi_{e,s})}\sum_{k=0}^{2^p-1} \absv{\frac{t}{1-t\uro}} \frac{2^p+k}{2^{p+1}}\frac{2^{2e}}{2^{2p}}\nonumber \\
&=  \absv{f'(\xi_{e,s})\xi_{e,s} + f(\xi_{e,s})}\absv{\frac{t}{1-t\uro}} \frac{3}{4}(2^p-1)\frac{2^{2e}}{2^{2p}}\nonumber \\
&=  \absv{f'(\xi_{e,s})\xi_{e,s} + f(\xi_{e,s})}\absv{\frac{t}{1-t\uro}} \frac{3}{4}(1-u)\frac{2^{2e}}{2^{p}}\nonumber \\
& \leq \absv{f'(\xi_{e,s})\xi_{e,s} + f(\xi_{e,s})} \frac{3}{4}\frac{2^{2e}}{2^{p}}\label{eq:error_term}
\end{align}
since $\absv{t}\leq 1$.

We now consider the case where $\frac{1}{2}<\absv{t}\leq 1$.  We can follow the derivation leading to  \cref{eq:case_one_half} by simply appending a coefficient given by $\one_{\fintvl[z]}$ at every step up to the penultimate one.  Now using \cref{lem:trange} and substituting $k$ by its value in terms of $z$ via $z=2^e(1+\frac{k}{2^p})$ we get:
\[
\one_{\fintvl[z]}\left(\frac{z(s,e,k)}{1-t\uro}\right) = 1\Leftrightarrow
\begin{cases}
 z\leq 2^e(\frac{1}{t}-\uro) & z> 0\\
 z\leq 2^e(-\frac{1}{t}+\uro) & z <0.
\end{cases}
\]
This means that in the last step of the derivation of  \cref{eq:case_one_half} we simply need to replace the bounds of the integrals by $2^e(\frac{1}{t}-\uro)$ and $2^e(\frac{1}{t}+\uro)$, which leads to the final expression
\[
\frac{1}{(1-t\uro)}\sum\limits_{s, e=\emin+1}^{\emax-1} \int_{(-1)^s2^e(1-\uro)}^{(-1)^s2^e(\frac{1}{\absv{t}}-\uro)}\frac{\absv{x}}{2^{e+1}} f(x) ~dx 
\]

In terms of error, it is clear that since we're approximating the same integrals $\int xf(x)~dx$ in the same way, but on a more narrow interval, the error term \cref{eq:error_term} also provides an upper bound the error made in the integrals $\int_{(-1)^s2^e(1-\uro)}^{(-1)^s2^e(\frac{1}{\absv{t}}-\uro)}\frac{\absv{x}}{2^{e+1}} f(x) ~dx$,  in other words 
\[
\absv{S(e,s,t)}\leq  \absv{f'(\xi_e)\xi_e + f(\xi_e)} \frac{3}{4}\frac{2^{2e}}{2^{p}}
\]
also when $\nicefrac{1}{2}\leq \absv{t}\leq 1$. It thus follows by the triangular inequality that we can bound the total error made from converting discrete sums into integrals by
\[
\frac{3}{4} \left(\sum_{s, \emin<e<\emax} \left(f'(\xi_{e,s})\xi_{e,s} + f(\xi_{e,s})\right)\frac{2^{2e}}{2^{p}}\right)
\]
as claimed.
\end{myproof}


\subsection*{Proofs for \cref{subsec:typical}.}

\begin{myproof}{\cref{thm:typical_dist}}
We start with the exact expression given in \cref{eq:LPerrorDensity}, namely
\begin{align*}
d(t)=\sum_{z\in\F\setminus\{-\infty,0,\infty\}}\one_{\fintvl[z]}\left(\frac{z}{1-t\uro}\right) f\left(\frac{z}{1-t\uro}\right) \frac{\uro\absv{z}}{(1-t\uro)^2}
\end{align*}
We are now going to sum over significands (since they are assumed to be equiprobable). Using \cref{lem:Ccoeff} to re-write $u\absv{z}$ and ignoring the terms with maximal exponents as we did in the proof of \cref{thm:HP_errorDensity} (this is not strictly necessary, but it make the proof a lot less cumbersome) we get:
\begin{align}
d(t) =&\sum_{k=0}^{2^p-1}\frac{C(e,k)}{(1-tu)^2}\sum_{e=emin+1}^{emax-1}\sum_{s=0}^1 f\left(\frac{z(e,s,k)}{1-tu}\right)(\ceil{z(e,s,k)}-\floor{z(e,s,k)})\nonumber
\\
=&\sum_{k=0}^{2^p-1}\frac{C(e,k)}{(1-tu)^2}\Pro{\round(X)=z(s,e,k)})+ S(k,p, t)\nonumber 
\\
&=\frac{1}{2^p(1-tu)^2}\left(\frac{2}{3}+\sum_{k=1}^{2^p-1}\frac{2^p+k}{2^{p+1}}\right)+ S(k,p,t) \nonumber
\\
&=\frac{1}{2^p(1-tu)^2}\left(\frac{2}{3}+\frac{3(2^{p}-1)}{4}\right)+ S(k,p,t)\label{eq:pdfleq1/2}
\end{align}
where $S(k,p,t)$  is an error term quantifying the error made by the numerical integration scheme evaluating $\Pro{\round(X)=z(s,e,k)})$ as the sum in the first step of the derivation above. It can be quantified in the same way as in the proof of \cref{thm:HP_errorDensity}. Crucially $\lim_{p\to\infty}S(k,p,t)=0$ for every $k,t$. 
We thus get that for $\absv{t}<\nicefrac{1}{2}$
\[
\lim_{p\to\infty}d(t) = \lim_{p\to\infty} \frac{1}{2^p(1-t2^{-p})^2}\left(\frac{2}{3}+\frac{3(2^{p}-1)}{4}\right)+ S(k,p) = \frac{3}{4}.
\]

As shown by \cref{lem:trange}, when $\frac{1}{2}< t\leq 1$  not all mantissas in the sum of \cref{eq:LPerrorDensity} are possible,. Using the explicit characterisation of \cref{lem:trange}, we get that for $\nicefrac{1}{2}< \absv{t}\leq 1$ \eqref{eq:pdfleq1/2} becomes 
\begin{align*}
d(t)&=\frac{1}{2^p(1-tu)^2}\left(\frac{2}{3}+\sum_{k=1}^{\alpha(t)}\frac{2^p+k}{2^{p+1}}\right) + S(k,p,t)
\\
&=\frac{1}{2^p(1-tu)^2}\left(\frac{2}{3}+\frac{1}{2}\alpha(t)+\frac{1}{2^{p+2}}(\alpha(t)^2)\right)+ S(k,p,t)
\end{align*}
where $\alpha(t)=\floor{2^p(\frac{1}{t}-1)-\frac{1}{2}}$ is the usual floor function applied to $2^p(\frac{1}{t}-1)-\frac{1}{2}$.  Taking the limit of the equation above we get
\begin{align}
\lim_{p\to\infty}d(t) & = \lim_{p\to\infty}  \frac{1}{(1-t2^{-p})^2}\left(\frac{1}{2}\left(\frac{1}{t}-1\right)+\frac{1}{4}\left(\frac{1}{t}-1\right)^2\right)+S(k,p,t)\nonumber \\
&= \frac{1}{2}\left(\frac{1}{t}-1\right)+\frac{1}{4}\left(\frac{1}{t}-1\right)^2\label{eq:pdfgeq1/2}
\end{align}
The case where $-1\leq t< -\frac{1}{2}$ is treated in the same way and yields the same asymptotic distribution.  Combining \ref{eq:pdfleq1/2} and \ref{eq:pdfgeq1/2} we get
\begin{align*}
\lim_{p\to\infty}d(t) =\begin{cases}
\frac{3}{4}&\text{if }t\in\left[-\frac{1}{2} ,\frac{1}{2}\right]  \\
\frac{1}{2}\left(\frac{1}{t}-1\right)+\frac{1}{4}\left(\frac{1}{t}-1\right)^2& \text{else }\\
\end{cases}
\end{align*}
as announced.
\end{myproof}

\subsection*{Proofs for \cref{subsec:covar}.}

We assume throughout this section that $\erel(X)$ is distributed according to $d_{hp}$ in \cref{eq:HPerrorDensity}.
We start by bounding $\Exp{\erel(X)}$. The proof of the following Proposition is a little technical but not conceptually difficult and can be found in the Appendix.

\begin{proposition}\label{prop:ExpError}
$K\frac{\uro}{6}\leq \Exp{\erel(X)}\leq K\frac{4\uro}{3}$
where
\[
K=\sum_{s,e=\emin+1}^{\emax-1} \int_{(-1)^s2^e(1-\uro)}^{(-1)^s2^e(2-u)} \frac{\absv{x}}{2^{e+1}} f(x) ~dx
\]
\end{proposition}
\begin{proof}
Consider the definition \cref{eq:HPerrorDensity}, when $\frac{1}{2}< \absv{t}\leq 1$. Note that the terms
\[
K(t)\defeq \sum\limits_{s, e=\emin+1}^{\emax-1} \int_{(-1)^s2^e(1-\uro)}^{(-1)^s2^e(\frac{1}{\absv{t}}-\uro)} \frac{\absv{x}}{2^{e+1}} f(x) ~dx
\]
are symmetric around 0 due to the dependence on $\absv{t}$ (note also that the constant $K= K(\frac{1}{2})$). It follows by a simple change of variable that
\begin{align*} 
\int_{-1}^{-\frac{1}{2}} \frac{t}{(1-t\uro)^2}K(t)~dt &= \int_{1}^{\frac{1}{2}}\frac{-s}{(1+s\uro)^2}K(-s)~(-ds)\\
&= \int_{\frac{1}{2}}^1 \frac{-s}{(1+s\uro)^2}K(s)~ds
\end{align*}
and thus
\begin{align*}
\int_{-1}^{-\frac{1}{2}} d(t)t~dt + \int^{1}_{\frac{1}{2}} d(t)t~dt& = -\int_{\frac{1}{2}}^{1}\frac{t}{(1+t\uro)^2} K(t)~dt +  \int^{1}_{\frac{1}{2}} d(t)t~dt
\\
& \geq - \int_{\frac{1}{2}}^{1}\frac{t}{(1-t\uro)^2} K(t)~dt + \int^{1}_{\frac{1}{2}} d(t)t~dt = 0
\end{align*}
From this it follows that the net contribution $\Exp{\erel(X)}$ of the two `wings' of the distribution is positive and in particular that
\[
\int_{-\frac{1}{2}}^{\frac{1}{2}}d(t) t~dt\leq \int_{-1}^{1}d(t) t~dt=\Exp{\erel(X)}.
\]
Moreover, by definition of $K$ we have
\[
\int_{-\frac{1}{2}}^{\frac{1}{2}}d(t) t~dt = K\int_{-\frac{1}{2}}^{\frac{1}{2}}\frac{t}{(1-t\uro)^2}~dt
\]
We now integrate by parts to get
\begin{align*}
K\int_{-\frac{1}{2}}^{\frac{1}{2}}\frac{t}{(1-t\uro)^2}~dt &= K \left.\frac{t}{\uro(1-t\uro)}\right|^{\frac{1}{2}}_{-\frac{1}{2}} - \int_{-\frac{1}{2}}^{\frac{1}{2}} \frac{1}{\uro(1-t\uro)}~dt
\\
&=\frac{K}{\uro}\left.\left(\frac{t}{\uro(1-t\uro)}-\frac{\ln(1-t\uro)}{\uro}\right)\right|^{\frac{1}{2}}_{-\frac{1}{2}}
\\
&=\frac{K}{\uro}\left(\frac{1}{1-\frac{\uro^2}{4}}- \frac{2}{\uro}\arctanh\left(\frac{\uro}{2}\right) \right)
\end{align*} 
For small values of $\uro$ the term above is very nearly linear with slope $\frac{1}{6}$ and intercept 0. To an extremely good approximation we therefore get for small values of $\uro$ (say below $2^{-2}$) that
\[
K\int_{-\frac{1}{2}}^{\frac{1}{2}}\frac{t}{(1-t\uro)^2}~dt = K\frac{\uro}{6}
\]
which proves the lower bound.

For the upper bound we proceed in a similar way. Note that 
\begin{align*}
\Exp{\erel(X)} =  &\int_{-1}^{-\frac{1}{2}}d(t) t~dt + \int_{-\frac{1}{2}}^{\frac{1}{2}}d(t) t~dt +\int_{\frac{1}{2}}^1d(t) t~dt
\\
= &
~\int_{-1}^{-\frac{1}{2}}\frac{t}{(1-t\uro)^2}K(t) ~dt +
K \int_{-\frac{1}{2}}^{\frac{1}{2}}\frac{t}{(1-t\uro)^2} ~dt + \int_{\frac{1}{2}}^1\frac{t}{(1-t\uro)^2} K(t) ~dt
\\
\leq & ~K \int_{-1}^{1}\frac{t}{(1-t\uro)^2}~dt\qquad\qquad\qquad\text{(since }K(t)\leq K\text{ )}
\\
= & ~ \frac{K}{\uro} \left(\frac{2}{(1-\uro^2)} - \frac{2}{\uro} \arctanh(\uro) \right)
\end{align*}
Again, the expression above is very nearly linear in $\uro$ for small values of $\uro$ (say below $2^{-2}$), with slope $\frac{4}{3}$ and intercept 0. It follows that for small values of $\uro$
\[
\Exp{\erel(X)}\leq K\frac{4\uro}{3}
\]
proving the upper bound.
\flushright{$\square$}
\end{proof}
Note that $K\leq 1$ since $\absv{x}\leq 2^{e+1}$ in each summand. The \emph{expected relative error $\Exp{\erel(X)}$ will therefore always be at most of the order of $\uro$} by the previous result. This makes good intuitive sense.

We now turn to $\Exp{X.\erel(X)}$. The proof of the following result is conceptually more involved. It is based on the fundamental definition of Lebesgue integration as the limit of a monotone sequence of so-called simple functions. 

\begin{proposition}\label{prop:expXerelX}$\Exp{X.\erel(X)} = \sum_{s,e} f((-1)^s2^e)(-1)^s2^{2e}\frac{3\uro^2}{2}$
\end{proposition}
In particular $\Exp{X.\erel(X)}=0$ if $f$ is symmetric. More generally, the value of $\Exp{X.\erel(X)}$ is determined by the competing terms $2^{2e}$ and $\uro^2$. A quick calculation shows that as long as the distribution $f$ assigns most of its mass to values below $2^{\frac{p}{2}}$, $\Exp{X.\erel(X)}$ will be of order $u$. In fact, for all the benchmarks considered in \cref{sec:evaluation} bar one, $\Exp{X.\erel(X)}$ is of the order of $\uro$ or smaller.
\begin{proof}
In order to (over)approximate the support of the joint distribution $\mathbb{P}$ and the integral \cref{eq:expdef}, we consider for each $n\in\N$ the following collection of sets
\begin{align*}
A^n(z,k) \defeq &\left[\floor{z}+\frac{k}{n}\tau(z), \floor{z}+\frac{k+1}{n}\tau(z)\right]\times \\
&\left[\erel\left(\floor{z}+\frac{k}{n}\tau(z)\right), \erel\left(\floor{z}+\frac{k+1}{n}\tau(z)\right)\right]
\end{align*}
where $z\in \F, 0\leq k < n$, and $\tau(z)\defeq \ceil{z}-\floor{z}$ is the size of the interval $\fintvl$. These sets contain pairs $(x,\erel(x))$ where $x$ is restricted to a sub-interval of $\fintvl$ whose size is controlled by $n$. For any $n$, the support of $\mathbb{P}$ is included in the union $\bigcup_{z,k}A^n(z,k)$, by definition of $\mathbb{P}$. Moreover,  by definition of the joint probability $\mathbb{P}$ and the assumption that $f$ is constant on $\fintvl$, we have
\begin{align*}
\Pro{A^n(z,k)} = & \int_{\floor{z}+\frac{k}{n}\tau(z)}^{\floor{z}+\frac{k+1}{n}} f(x)~dx\\
= & f(z) \frac{\tau(z)}{n}
\end{align*}
The sets $A^n(k,z)$ were chosen is such a way that the \emph{simple functions} (see \cite{dudley2002real}) $h_n: \R\times [-1,1]\to\R$ given by
\[
h_n(x,t) = \sum_{z,k} \sup\{y\uro s : (y,s)\in A^n(z,k)\} .\one_{A^n(z,k)}(x,t)
\]
provide a monotonically decreasing (over)approximation of the function
\[
h(x,t) = x\uro t
\]
which we want to integrate. Using the common notation of integration theory: $h_n \downarrow h$. It now follows from the definition of the Lebesgue integral that
\begin{align*}
\Exp{X.\erel(X)} & = \int_{\R\times [-1,1]} x\uro t~d\mathbb{P}\\
& \defeq \int_{\R\times [-1,1]} h(x,t)~d\mathbb{P}\\
&\leq \int_{\R\times [-1,1]} h_n(x,t)~d\mathbb{P}\\
&= \sum_{z,k} \sup_{(x,t)\in A^n(z,k)} x\uro t~ \Pro{A^n(z,k)}
\end{align*}
The supremum is easily seen to be
\[
\sup_{(x,t)\in A^n(z,k)} x\uro t = 
\begin{cases}
\left(\floor{z}+\frac{k+1}{n}\tau(z)\right)\erel(\floor{z}+\frac{k+1}{n}\tau(z)) & z > 0
\\
\left(\floor{z}+\frac{k}{n}\tau(z)\right)\erel(\floor{z}+\frac{k}{n}\tau(z)) & z<0
\end{cases}
\]
Clearly, multiplying any real by its own relative error will yield its absolute error:
\[
\sup_{(x,t)\in A^n(z,k)} x\uro t = 
\begin{cases}
(\floor{z}+\frac{k+1}{n}\tau(z))-z & z > 0\\
(\floor{z}+\frac{k}{n}\tau(z))-z & z<0
\end{cases}
\]
and we thus have
\begin{align*}
\Exp{X.\erel(X)}\leq & \sum_{k,z>0} (\floor{z}+\frac{k+1}{n}\tau(z))-z) f(z) \frac{\tau(z)}{n} + \\
&\sum_{k,z<0} (\floor{z}+\frac{k}{n}\tau(z))-z) f(z) \frac{\tau(z)}{n}\\
 = &\sum_{z>0} f(z) \sum_k (\floor{z}+\frac{k+1}{n}\tau(z))-z)\frac{\tau(z)}{n}+\\
&\sum_{z<0} f(z) \sum_k (\floor{z}+\frac{k}{n}\tau(z))-z)\frac{\tau(z)}{n}.
\end{align*}
It is not too hard to see that as $n\to\infty$ the expression $\sum_k (\floor{z}+\frac{k+1}{n}\tau(z))-z)\frac{\tau(z)}{n}$ and $\sum_k (\floor{z}+\frac{k}{n}\tau(z))-z)\frac{\tau(z)}{n}$ both approach the Riemann integral
\[
\int_{\floor{z}}^{\ceil{z}} (x-z)~dx
\] 
In other words:
\[
\Exp{X.\erel(X)}\leq \sum_z f(z) \int_{\floor{z}}^{\ceil{z}} (x-z)~dx
\]
There are now two cases. If the significand of $z$ is zero then the interval $\fintvl$ is not symmetric around $z$, it is in fact divided in a $\frac{1}{3}$:$\frac{2}{3}$ ratio around $z$ and we therefore have
\begin{align*}
\int_{\floor{z}}^{\ceil{z}} (x-z)~dx &= \int_{\floor{z}}^{z+\frac{\tau(z)}{3}} (x-z)~dx+ \int_{z+\frac{\tau(z)}{3}}^{\ceil{z}} (x-z)~dx\\
&=0+\int_{z+\frac{\tau(z)}{3}}^{\ceil{z}} (x-z)~dx\\
\end{align*} 
For $z=z(0,e,0)=2^e$ the expression above yields:
\[
\int_{2^e(1+\uro)}^{2^e(1+2\uro)} (x-2^e)~dx = 2^{2e}\frac{3\uro^2}{2}.
\]
Similarly for $z=z(1,e,0)=-2^e$ we get
\[
\int^{-2^e(1+\uro)}_{-2^e(1+2\uro)} (x+2^e)~dx = -2^{2e}\frac{3\uro^2}{2}.
\]
All other intervals $\fintvl$ are centred around $z$ and therefore
\[
\int_{\floor{z}}^{\ceil{z}} (x-z)~dx = 0
\]
We thus have the following bound for the expectation $\Exp{X.\erel(X)}$:
\[
\Exp{X.\erel(X)}\leq \sum_{s,e} f((-1)^s2^e)(-1)^s2^{2e}\frac{3\uro^2}{2}
\]
A completely analogous argument using an under-approximating sequence of simple function $h_n\uparrow h$ defined by $\inf(A^n(z,k))$ instead of $\sup(A^n(z,k))$ shows that
\[
\Exp{X.\erel(X)}\geq \sum_{s,e} f((-1)^s2^e)(-1)^s2^{2e}\frac{3\uro^2}{2}
\]
and the equality follows.
\flushright{$\square$}
\end{proof}



\begin{myproof}{\cref{thm:covar}}
This is a direct corollary of \cref{prop:ExpError} and \cref{prop:expXerelX}.
\end{myproof}

%
%\begin{align*}
%\int_{-1}^1 d(t) t~dt = \underbrace{\int_{-1}^{-\frac{1}{2}}d(t) t~dt}_A + \underbrace{\int_{-\frac{1}{2}}^{\frac{1}{2}}d(t) t~dt}_B +\underbrace{\int_{\frac{1}{2}}^1d(t) t~dt}_C
%\end{align*}
%
%
%For notational clarity we write 
%\begin{align*}
%K & = \sum\limits_{s, e=\emin+1}^{\emax-1} 2^{e}u f(z(s,e,0))  \\
%L & = \sum\limits_{s, e=\emin+1}^{\emax-1} \int_{(-1)^s2^e(1+u)}^{(-1)^s2^e(2-u)} \frac{\absv{x}}{2^{e+1}} f(x) ~dx.
%\end{align*}
%We now integrate by parts to get
%\begin{align*}
%B & = \int_{-\frac{1}{2}}^{\frac{1}{2}} \frac{(K+L) t}{(1-tu)^2}~dt\\
%& = \frac{(K+L)}{u}\left.\left(\frac{t}{1-tu}+\frac{\ln(1-tu)}{u}\right)\right|^{\frac{1}{2}}_{-\frac{1}{2}}\\
%&=\frac{(K+L)}{u}\left(\frac{1}{1-\frac{u^2}{4}}- 2\arctanh\left(\frac{u}{2}\right) \right)
%\end{align*}
%For small values of $u$ the term in $u$ is almost linear  with slope $\frac{1}{6}$ and intercept 0. To a very good approximation we get for small $u$
%\[
%B = (K+L)\frac{u}{6}
%\]
%Now for the terms $A$ and $C$ which are integrated similarly.
%\begin{align*}
%C & = \int_{\frac{1}{2}}^1 \frac{t}{(1-tu)^2}\sum_{s, e=\emin+1}^{\emax-1} \left(2^{e}u f(z(s,e,0)) + \int_{(-1)^s2^e(1+u)}^{(-1)^s2^e(\frac{1}{\absv{t}}-u)} \frac{\absv{x}}{2^{e+1}} f(x) ~dx \right)~dt
%\end{align*}
%Using the definition of $K$ given above we get
%\begin{align*}
%A &= \int^{-\frac{1}{2}}_{-1} \frac{Kt}{(1-tu)^2} ~dt + \sum_{s, e=\emin+1}^{\emax-1}  \int^{-\frac{1}{2}}_{-1} \frac{t}{(1-tu)^2}\int_{(-1)^s2^e(1+u)}^{(-1)^s2^e(\frac{1}{\absv{t}}-u)} \frac{\absv{x}}{2^{e+1}} f(x) ~dx ~dt
%\\
%C &= \int_{\frac{1}{2}}^1 \frac{Kt}{(1-tu)^2} ~dt + \sum_{s, e=\emin+1}^{\emax-1}  \int_{\frac{1}{2}}^1 \frac{t}{(1-tu)^2}\int_{(-1)^s2^e(1+u)}^{(-1)^s2^e(\frac{1}{\absv{t}}-u)} \frac{\absv{x}}{2^{e+1}} f(x) ~dx ~dt
%\end{align*}
%The first summands of $A$ and $C$ are evaluated just like $B$ and we get
%\begin{align}
%&\frac{K}{u}\left.\left(\frac{t}{1-tu}+\frac{\ln(1-tu)}{u}\right)\right|_{-1}^{-\frac{1}{2}}\nonumber \\
%=&
%\frac{K}{u}\left(\frac{1}{2(1+u)(1+\frac{u}{2})}+\frac{1}{u}\ln\left(\frac{1+\frac{u}{2}}{1+u}\right)\right)\label{eq:app:A1}
%\end{align}
%and
%\begin{align}
%&\frac{K}{u}\left.\left(\frac{t}{1-tu}+\frac{\ln(1-tu)}{u}\right)\right|^{1}_{\frac{1}{2}}\nonumber \\
%=&
%\frac{K}{u}\left(\frac{1}{2(1-u)(1-\frac{u}{2})}+\frac{1}{u}\ln\left(\frac{1-u}{1-\frac{u}{2}}\right)\right)\label{eq:app:C1}
%\end{align}
%Again, the expression \cref{eq:app:A1} (resp. \cref{eq:app:C1}) is nearly linear for small values of $u$ with intercept $-\frac{3}{8}$ (resp. $\frac{3}{8}$) and slope $\frac{7}{12}$. The sum of \cref{eq:app:A1} and \cref{eq:app:C1} is thus
%\[
%\left(-\frac{3}{8}+\frac{7u}{12}\right)K + \left(\frac{3}{8}+\frac{7u}{12}\right)K = \frac{7u}{6}K
%\]
%Since we know the antiderivative of $\frac{t}{(1-tu)^2}$, we can compute the double integral in the definition of $A$ and $C$ above via an integration by part and the fundamental theorem of calculus. More precisely we get
%\begin{align}
%&\sum_{s,e=\emin+1}^{\emax-1}\int^{-\frac{1}{2}}_{-1} \frac{t}{(1-tu)^2}\int_{(-1)^s2^e(1+u)}^{(-1)^s2^e(\frac{1}{\absv{t}}-u)} \frac{\absv{x}}{2^{e+1}} f(x) ~dx ~dt\nonumber \\
%=&\sum_{s,e=\emin+1}^{\emax-1}\left.\frac{1}{u}\left(\frac{t}{1-tu}+\frac{\ln(1-tu)}{u}\right)\int_{(-1)^s2^e(1+u)}^{(-1)^s2^e(\frac{1}{\absv{t}}-u)}\frac{\absv{x}}{2^{e+1}}f(x)~dx\right|_{-1}^{-\frac{1}{2}} - 
%\nonumber \\
%& \sum_{s,e=\emin+1}^{\emax-1}\int^{-\frac{1}{2}}_{-1}  \frac{1}{u}\left(\frac{t}{1-tu}+\frac{\ln(1-tu)}{u}\right) \frac{2^e(\frac{1}{t}-u)}{2^{e+1}}f\left((-1)^s2^e\left(\frac{1}{t}-u\right)\right) (-1)^s2^e\ln(t)~dt\label{eq:app:A2}
%\end{align}
%and
%\begin{align}
%&\sum_{s,e=\emin+1}^{\emax-1}\int_{\frac{1}{2}}^1 \frac{t}{(1-tu)^2}\int_{(-1)^s2^e(1+u)}^{(-1)^s2^e(\frac{1}{t}-u)} \frac{\absv{x}}{2^{e+1}} f(x) ~dx ~dt\nonumber \\
%=&\sum_{s,e=\emin+1}^{\emax-1}\left.\frac{1}{u}\left(\frac{t}{1-tu}+\frac{\ln(1-tu)}{u}\right)\int_{(-1)^s2^e(1+u)}^{(-1)^s2^e(\frac{1}{t}-u)}\frac{\absv{x}}{2^{e+1}}f(x)~dx\right|^{1}_{\frac{1}{2}} - 
%\nonumber \\
%& \sum_{s,e=\emin+1}^{\emax-1}\int_{\frac{1}{2}}^1  \frac{1}{u}\left(\frac{t}{1-tu}+\frac{\ln(1-tu)}{u}\right) \frac{2^e(\frac{1}{t}-u)}{2^{e+1}}f\left((-1)^s2^e\left(\frac{1}{t}-u\right)\right) (-1)^s2^e\ln(t)~dt\label{eq:app:C2}
%\end{align}
%The first summand of \cref{eq:app:C2}  evaluates to
%\begin{align}
%\sum_{s,e=\emin+1}^{\emax-1}-\frac{1}{u}\left(\frac{1}{1-u}+\frac{\ln(1-u)}{u}\right)\int_{(-1)^s2^e(1-u)}^{(-1)^s2^e(1+u)}\frac{\absv{x}}{2^{e+1}}f(x)~dx
%\nonumber \\
%\hspace{1.3em} -\frac{1}{u}\left(\frac{1}{2(1-\frac{u}{2})}+\frac{\ln(1-\frac{u}{2})}{u}\right) 
%\int_{(-1)^s2^e(1+u)}^{(-1)^s2^e(2-u)}\frac{\absv{x}}{2^{e+1}}f(x)~dx \label{eq:app:C22}
%\end{align}
%The function in from of the first integral is almost linear for small values of $u$ with intercept $\frac{1}{8}$ and slope $\frac{1}{12}$. Similarly, the function in front of the second integral is almost linear for small values of $u$ with intercept $\frac{1}{2}$ and slope $\frac{2}{3}$. Moreover, under \cref{ass:regularity} the last integral can be re-written as
%\[
%\int_{(-1)^s2^e(1-u)}^{(-1)^s2^e(1+u)}\frac{\absv{x}}{2^{e+1}}f(x)~dx =\frac{2^e}{2^{e+1}} f(z(s,e,0) 2^e u=\frac{1}{2} 2^e u f(s,e,0).
%\] 
%Using the definitions of $K$ and $L$ we can now re-write \cref{eq:app:C22} as
%\[
%\left(\frac{1}{2}+\frac{2u}{3}\right)\frac{K}{2}+\left(\frac{1}{8}+\frac{u}{12}\right)L
%\]
%A similar computation for the first summand of \cref{eq:app:A2} yields
%\[
%\left(\frac{1}{2}-\frac{2u}{3}\right)\frac{K}{2}+\left(\frac{1}{8}-\frac{u}{12}\right)L
%\]
%and as a result the sum of the first summands in \cref{eq:app:A2} and \cref{eq:app:C2} yield
%\[
%\frac{K}{4}+\frac{L}{8}
%\]

%Variance:
%
%\begin{align*}
%\int_{-1}^1d(t)t^2~dt = \underbrace{\int_{-1}^{-\frac{1}{2}}d(t) t^2~dt}_A + \underbrace{\int_{-\frac{1}{2}}^{\frac{1}{2}}d(t) t^2~dt}_B +\underbrace{\int_{\frac{1}{2}}^1d(t) t^2~dt}_C
%\end{align*}
%We integrate by substitution by putting $x=1-tu$ and we get
%\begin{align*}
%B & = \int_{-\frac{1}{2}}^{\frac{1}{2}} \frac{(K+L) t^2}{(1-tu)^2}~dt\\
%\int_{-\frac{1}{2}}^{\frac{1}{2}} \frac{(K+L) t^2}{(1-tu)^2}~dt\\
%& = \frac{(K+L)}{u}\left.\left(\frac{t}{1-tu}+\frac{\ln(1-tu)}{u}\right)\right|^{\frac{1}{2}}_{-\frac{1}{2}}\\
%&=\frac{(K+L)}{u}\left(\frac{1}{1-\frac{u^2}{4}}- 2\arctanh\left(\frac{u}{2}\right) \right)
%\end{align*}

\section{Implementation}\label{sec:appdx-implementation}

\subsubsection{Independent Operation.}\label{sec:appdx-independent}
We report the pseudo-code in \cref{independentalg}.
%
When the operands $X,Y$ are independent, we can compute $Z=X\iop Y$ with a simple pairwise operation (line 3, 4) between focal elements using interval arithmetic (line 5), and we multiply the corresponding probabilities (line 6).  
%
No matter the independence, we still populate and propagate the trace for future dependent operations (line 8).
%
This is very much similar to the pen-and-paper approach using a joint table.
% 
\begin{algorithm}[H]
	\caption{Independent Operation $Z=X\iop Y$}\label{independentalg}
	\begin{algorithmic}[1]
		\Function {ind\_op}{$DS_X,\iop,DS_Y,trace_x,trace_y$} 
		\State $DS_Z=list()$
		\ForAll {$ ([x_1, x_2], p_{x}) \in DS_X$} 
		\ForAll {$ ([y_1, y_2], p_{y}) \in DS_Y$}
		\State $[z_1, z_2]=[x_1, x_2]\;op\;[y_1, y_2]$ \Comment{operation between intervals}
		\State $p_{z}=p_{x}*p_{y}$
		\State $DS_Z.append(([z_1, z_2],p_z))$
		\EndFor
		\EndFor
		\State $trace_Z=trace_X \cup trace_Y\cup\{Z=X\iop Y\}$
		\State \textbf{return} $DS_Z$, $trace_Z$
		\EndFunction
	\end{algorithmic}
\end{algorithm}

\subsection{Linear Programming Routine}\label{sec:appdx-lp}
We can create a linear programming (LP) routine to delineate upper bound and lower bounds for the p-box.
%
In the following we report the pseudo-code for the evaluation of the upper bound of a p-box. The lower bound is similar.
%

\begin{align*}
%1\leq i,j\leq N \land 
&\text{\textbf{maximize}}\sum_{evp\:\in\:z_{i,j}} p_{z_{i,j}}\\
&\text{\textbf{subject to}}\quad 0\leq p_{z_{i,j}}\leq 1\\
&\qquad\qquad\quad\quad \forall i \in [1,N], \sum_{1\leq j\leq N} p_{z_{i,j}} = p_{y_{i}} \\
&\qquad\qquad\quad\quad \forall j \in [1,N], \sum_{1\leq i\leq N} p_{z_{i,j}} = p_{x_{i}}\\
\end{align*}
%


%
The probabilities in the DS structures of the operands ($p_{y_i}$, $p_{x_i}$) are called the \emph{marginals}. The focal elements we use to populate $DS_Z$ are called the \emph{insiders} ($p_{z_{i,j}}$).
%
The constraints in the LP program force the insiders to have a probability between 0 and 1, and they relate the insiders with the marginals. The insiders have to sum up to the marginals. This is very similar to a so called \emph{joint table}.
%

The LP program takes in input an \emph{evaluation point} (evp) and returns a probability value.
%
In order to construct the $DS_Z$ exactly we should pick one evaluation point per focal element. This has quadratic complexity.
%
A good trade-off between accuracy and execution time consists in picking only $N$ out of the $N^2$ evaluation points (e.g. using some heuristics), at the price of a slightly over-approximation.
%
We run the linear programming routines in parallel, because the analysis of a single evaluation point is completely independent from the others.
%

\section{Experimental Evaluation}
\label{sec:appdx-evaluation}


\begin{table*}[tb]
	\scriptsize
	\caption{Comparison between \Tool, given uniform (uni), normal (norm) and exponential (exp) input distributions, and FPTaylor. The PAF columns report 99\% of the support of the output range distribution. The FPTaylor columns report the worst-case output ranges. The asterisk (*) highlights a difference of more than one order of magnitude between \Tool and FPTaylor.}
	%\caption{ All input variables are distributed according to truncated normal distributions. Column \emph{Time} is the execution time of \Tool in seconds, \emph{Range Distribution} compares the CDFs of the computed output ranges, and \emph{Absolute Error Distribution} compares the CDFs of the absolute errors. Bold font denotes the best results.}
    \label{rangeanalysis}
	\renewcommand{\arraystretch}{1.1}
	\setlength{\tabcolsep}{0.1em} % for the horizontal padding
	\centering
	\begin{tabular}{@{\extracolsep{2.pt}}p{1.4cm}L{2.55cm}L{2.7cm}L{2.6cm}L{2.5cm}@{}}
		\toprule
		\multirow{1}{*}{Benchmark} &  \multicolumn{1}{c}{\Tool (uni)} &
		\multicolumn{1}{c}{\Tool (norm)} & \multicolumn{1}{c}{\Tool (exp)} &
		\multicolumn{1}{c}{FPTaylor} \\
		%\cmidrule{2-3} \cmidrule{4-5} \cmidrule{6-7} \cmidrule{8-9}
		%\multicolumn{1}{c}{}&\multicolumn{1}{c}{min} & \multicolumn{1}{c}{max} & \multicolumn{1}{c}{min} &\multicolumn{1}{c}{max}& \multicolumn{1}{c}{min} & \multicolumn{1}{c}{max} & \multicolumn{1}{c}{min} & \multicolumn{1}{c}{max}\\
		%\cmidrule{3-4} \cmidrule{5-6} \cmidrule{7-8} \cmidrule{9-10}
		%& & \multicolumn{1}{c}{Time} & \multicolumn{1}{c}{Range} & \multicolumn{1}{c}{\Tool} & \multicolumn{1}{c}{Samp} &\multicolumn{1}{c}{\Tool} & \multicolumn{1}{c}{Samp} & \multicolumn{1}{c}{\Tool} & \multicolumn{1}{c}{Samp} \\
		\midrule
		%bspline0 & -0.00000000077883455679418935337784162753219851409744 & 0.16666672627132032385156953153389113321064103956823 & -0.00000000058492024915427649994717652305480669871684 & 0.16666672627132032385156953153389113321064103956823 & -5.72375381165577332956e-08 & 1.66666723904204810980e-01\\
		bspline0 & [-5.85e-08, 1.67e-01] & [-5.85e-08, 1.67e-01] & [1.22e-01, 1.67e-01]* & [-5.72e-08, 1.67e-01]\\
		\mydashline{}
		%bspline1 & 0.16520278614663562686258977594393792590438326867958 & 0.66678881744887807026678482276263451108329832734717 & 0.64745996452478977890933915367684718516053428361777,0.66678881744887807026678482276263451108329832734717 &  1.55039038364818776428e-01 & 6.74479359284310686640e-01\\
		bspline1 & [1.65e-01, 6.67e-01] & [1.65e-01, 6.67e-01] & [6.47e-01, 6.67e-01] & [1.55e-01, 6.75e-01]\\
		\mydashline{}
		%bspline2 & 0.16654457214215405014441068929351985072457598498551 & 0.66736971318116047348004362723777682854869515868838 & 0.16654457 & 0.6673697 & 1.5885395660e-01 & , 6.782943124089e-01\\
		bspline2 & [1.67e-01, 6.67e-01] & [1.67e-01, 6.68e-01] & [1.67e-01, 2.34e-01] & [1.59e-01, 6.78e-01] \\
		\mydashline{}
		%bspline3 & -0.1666667 & 0 & 0 & 0 & [-1.66666708886624159724e-01, 4.22199574560503934383e-08]\\
		bspline3 & [-1.67e-01, 5.22e-08] & [-1.67e-01, 5.22e-08] & [-4.04e-04, 5.22e-08]* & [-1.67e-01, 4.22e-08]\\
		\mydashline{}
		%classids0 & -23.19800395835005075489414566282034293364533745997787 & 24.73341184076822085638500490360238749207043399524884 & -23.19800395835005075489414566282034293364533745997787 & 7.225368818035918373096678 & -2.31980032317030513411e01 & 2.47334100508209679958e01\\
		classids0 & [-2.32e01, 2.47e01] & [-2.32e01, 7.22e00] & [-1.43e01, -8.93e-01]* & [-2.32e01, 2.47e01]\\
		\mydashline{}
		%classids1 & -10.19965712066813214936320834735423585244276331676875 & 12.04672317126114713466209485212356110879041956677215 & -0.27960376594320436744098980826831448709413052711171 & 5.18561888956570263693631491639079579717415645519398 & -1.02362301744647137269e01 & 1.20832962881602039573e01 \\
		classids1 & [-1.02e01, 1.20e01] & [-3.81e00, 1.21e01] & [-2.80e-01, 5.18e00]* & [-1.02e01, 1.21e01]\\
		\mydashline{}
		%classids2 & -16.72542125137536466667032051427444685731675538730276 & 13.34294792371044540626727529807171095643784913730281 & -9.79786621389407412874682180006851189894839250624711 & 13.34294792371044540626727529807171095643784913730281 & -1.67932948124692167369e01 &  1.34108218796558080754e01\\
		classids2 & [-1.67e01, 1.33e01] & [-9.80e00, 1.33e01] & [-8.06e-01, 8.91e00]* & [-1.68e01, 1.34e01]\\
		\mydashline{}
		%example & 0.0 & 110.0 & 0.0 & 8.08 & -1.764298e-5 & 1.100001e+2\\
		%example & 0.00e00 & 1.10e02 & 0.00e00 & 8.08e00 & & &-1.76e-5 & 1.10e02\\
		%\mydashline{}		
		%doppler1 & -143.25534736281833618702136234494938662464846856892253 & -0.03389187549027944022167141846401389599999999999999 & -7.55751153737632755738188069510571242141017764073517,-0.03389187385925553743827755861602349599999999999999 & -7.81e00 & -3.39e-02 & -1.38956168632608438429e02 & -3.16361067356379296878e-02\\
		doppler1 & [-1.43e02, -3.39e-02] & [-7.56e-02, -3.39e-02]* & [-7.81e00, -3.39e-02]* & [-1.39e02, -3.16e-02]\\
		\mydashline{}
		%doppler2 & -244.57727857257911355657451174785882699325156863779075 & -0.02267864139759156493714946257621464599999999999999 & -9.29055854258564986871407980356772719261351197314252 & -0.02267864139759156493714946257621464599999999999999 & 0 & [-2.323996e2, -2.076267e-2]\\
		doppler2 & [-2.45e02, -2.27e-02] & [-9.29e00, -2.27e-02]* & [-1.07e01, -2.27e-02]* & [-2.32e02, -2.08e-02] \\
		\mydashline{}
		%doppler3 & -86.14904871345917624168388176784579278316418776858484 & -0.50591693289022643184005773914955157999999999999999 & -7.97659988231432303095885400512473062291241062203234 & -0.50591693289022643184005773914955157999999999999999 & -7.38774278151304089612662572288497154892268909902438 & -0.50591693289022643184005773914955157999999999999999 & -8.347873e1 & -5.030512e-1\\
		doppler3 & [-8.62e-01, -5.05e-01] & [-7.98e00, -5.05e-01]* & [-7.39e00, -5.06e-01]* & [-8.35e01, -5.03e-01]\\
		\mydashline{}
		%filter1 & -1.40000014305114959259187257812341924623224057674412 & 1.40000014305114 & 0 & 0 & 0 & [-1.40000012516975491650e00, 1.40000012516975491650e00]\\
		filter1 & [-1.40e00, 1.40e00] & [-1.40e00, 1.40e00] & [-1.12e-01, 1.12e-01]* & [-1.40e00, 1.40e00]\\
		\mydashline{}
		%filter2 -2.04000090122231313231772256752436487497090466239063 & 2.04000090122231313231772256752436487497090466239010 & -2.04000090122231313231772256752436487497090466239063 & 2.04000090122231313231772256752436487497090466239010 & -2.05531329300255460879e00 & 2.05531329300255460879e00\\
		filter2 & [-2.04e00, 2.04e00] & [-2.04e00, -2.04e00] & [-9.54e-01, 8.81e-01]* & [-2.06e00, 2.06e00]\\
		\mydashline{}
		%filter3 & -2.36400287771284186582081697672070681707025012874979 & 2.36400287771284186 & -2.36400287771284186582081697672070681707025012874979 & 2.36400287771284186582081697672070681707025012875344 & -2.38970535224681102093e00 & 2.38970535224681102093e00\\
		filter3 & [-2.36e00, 2.36e00] & [-2.36e00, 2.36e00] & [-2.36e00, 2.36e00] & [-2.39e00, 2.39e00] \\
		\mydashline{}
		%filter4 & -3.26560528764886420662654955940711873378793418705969 & 3.26560528764886420662654955940711873378793418707367 & 0 & 0 & -3.32193193778378415004e00 & 3.32193193778378415004e00]\\
		filter4 & [-3.26e00, 3.26e00] & [-3.26e00, 3.26e00] & [-3.27e00, 3.27e00] & [-3.32e00, 3.32e00] \\
		\mydashline{}
		%rigidbody1 & -705.00028610234549480422370605604630680481600838671787 & 705.00028610234549480422370605604630680481600838671741 & -50.52000 & 56.880001744651743872 & 0 & [-7.05000158309937319245e02, 7.05000158309937319245e02]\\
		rigidbody1 & [-7.05e02, 7.05e02] & [-5.05e01, 5.69e01]* & [-4.45e01, 6.00e01]* & [-7.05e02, 7.05e02]\\
		\mydashline{}
		%rigidbody2 & -56010.18875426759551307199242278240600000000000000056011 & 58740.03476322624819144550713862034887943532128121560907 & -476.83884276093289492144289177401996206313675697600724 & 868.75218330328520882221963351098962683145641597353794 &-5.63283006129269924713e04 & 5.87400193629269924713e04\\
		rigidbody2 & [-5.60e04, 5.87e04] & [-4.77e03, 8.70e03]* & [-1.44e02, 3.74e02]* & [-5.63e04, 5.87e04] \\
		\mydashline{}
		%sine & -1.00065240254112604208311969069635444000000000000002 & 1.00054271272033551966911 & -1.00065240254112604208311969069635444000000000000002 & 1.00054271272033551966911 & -1.01511526142621688606e00 & 1.01511526142621688606e00\\
		sine & [-1.00e00, 1.00e00] & [-1.00e00, 1.00e00] & [-2.61e-01, 4.21e-01] & [-1.02e00, 1.02e00]\\
		%sineorder3 & 0 & 0 & 0 & 0 & 0 & [-1.01488840237639554864e00, 1.01488840237639554864e00]\\
		%sineorder3 & 0 & 0 & 0 & 0 & -1.02e00 & 1.02e00\\
		\mydashline{}
		%solvecubic & -0.92654074885795904201836782868366933916270620211473 & 48.14817199001138737937221142630697946904012539607078 & 0 & -0.92654074885795904201836782868366933916270620211473 & 9.86911411108404098244815445857033351373388541105679& -9.34079981944043069575e-01 & 4.81481641833329447877e01\\
		solvecubic & [-9.27e-01, 4.81e01] & [-9.27e-01, 1.37e01] & [-9.27e-01, 9.87e00] & [-9.34e-01, 4.82e01]\\
		\mydashline{}
		%sqrt & -334.64403326205426989752322150087244200000000000000335 & 1.48310329002878149835153054972962680000000000000002 & -0.75674285858787171893617435154900973781622042224627 & 1.48310329002878149835153054972962680000000000000002 & -3.37587425718559757115e02 & 1.50082009894056533916e00]\\
		sqrt & [-3.35e02, 1.48e00] & [-7.57e-01, 1.48e00]* & [-2.11e-01, 1.46e00]* & [-3.38e02, 1.50e00] \\
		\mydashline{}
		%traincars1 & -2672.86638757550976013776311716481173700832752592277681 & 5443.88224993657944054953659658779139241077870330061791 & 2031.86646706634881819903933981993562668203335585664643 & 5443.8822499365794 & 0 & [-2.67286669711423610352e03, 5.44388336543638433795e03]\\
		traincars1 & [-2.67e03, 5.44e03] & [2.03e03, 5.44e03]* & [3.36e03, 5.44e03]* & [-2.67e03, 5.44e03] \\
		\mydashline{}
		%traincars2 & -3645.75615496652165037134205497156537013609501699360242 & 3845.60407498433211736307987858967591470795179260403513 & -1389.95617298156458380623409598657550812879056051864932 & 1451.19814936196921017325521611070042543858295422126966 &-641.10524286363947517746900307008590428864519570325274 & 656.95963731567704517676443397603384314934147865849238 &-3.64575614506405099746e03 & 3.84560391406112285040e03\\
		traincars2 & [-3.65e03, 3.85e03] & [-1.39e03, 1.45e03] & [-6.41e02, 6.57e02] & [-3.65e03, 3.85e03] \\
		\mydashline{}
		%traincars3 & -66966.99626444690564611396442169338481123962812480991809,67314.42160718281628355119705936362399190849606777061014 & 0 & 0 & 0 & [-6.69669937956783542177e04, 6.73144182583322253777e04]\\
		traincars3 & [-6.70e04, 6.73e04] & [-2.52e04, 2.43e04] & [-1.42e04, 1.33e04] & [-6.70e04, 6.73e04] \\
		\mydashline{}
		%traincars4 & -584294.61602199649377425550548553843728601387510213591314 & 569108.82430793388380964770579996723320990805979206265441 & -232658.44691273444479503034688910392691174779978431085409 & 569108.82430793388380964770579996723320990805979206265441 & -5.84294571957051986828e05 & 5.69108790856380830519e05\\
		traincars4 & [-5.84e05, 5.69e05] & [-2.33e05, 5.69e05] & [-1.15e05, 1.27e05] & [-5.84e05, 5.69e05] \\
		\mydashline{}
		%trid1 & -12801.10326534499676105086320804507739438409455049229106 & 12601.07038144138010763519864549091540000000000000012602 & -554.56033653637315761418981072691451095849303431209528 & 771.60012977383425573949345887631521463820007752262909 & 0 & [-1.287909e4, 1.267909e4]\\
		trid1 & [-1.28e04, 1.26e04] & [-5.55e03, 7.72e03] & [-6.91e02, 7.68e02] & [-1.29e04, 1.27e04] \\
		\mydashline{}
		%trid2 & -32602.11128850179245324789693423050840000000000000032603 & 13343.05750458537837448654659717071420000000000000013344  & -1652.04043129278254252997031978364980621448525118001548 & 1590.07245531931941751345017347865116728135042030200815 & -1962.76012808198445520890276966039433204696454957657925 & 1811.79829860555382023438883298384903977819496738309141 & -3.268009e4 & 1.346905e4\\
		trid2 & [-3.26e04, 1.33e04] & [-1.65e03, 1.59e03]* & [-1.96e03, 1.81e03]* & [-3.27e04, 1.35e04]\\
		\mydashline{}
		%trid3 & -52803.12179133131679135897260900710789456176502645842130 & 14243.77153968691228095175930347230800000000000000014244 & -2977.279839 & 3949.12009716 & -4057.20068889796268599710913953772283374219712376542330 & 4266.79986987972102568322778781935169613808913505876627 & -5.295896e4 & 1.398170e4\\
		trid3 & [-5.28e04, 1.42e04] & [-2.98e03, 3.95e03]* & [-4.06e03, 4.27e03]* & [-5.30e04, 1.40e04] \\
		\mydashline{}
		%trid4 & -72604.13458310118310939369453627036600000000000000072605 & 19209.27746845405165423997389790578400000000000000019210 & -6096.24040850977282807576337087605565863940824133748067 & 11706.36891312462065902198308701537738024933859560416532 & -7.275997e4 &  1.544205e4\\
		trid4 & [-7.26e04, 1.92e04] & [-6.10e03, 1.17e04]* & [-8.13e03  1.05e04]* & [-7.28e04, 1.54e04]\\
		\bottomrule
	\end{tabular}
\end{table*}



%\input{tables/table-uniform-cdf}
%In Table~\ref{cdfuni} we report the absolute variational distance between the
%CDFs of \Tool and sampling with respect to the golden distribution, assuming
%that all the input variables are uniformly distributed.
%
%On average, the execution time of \Tool is 127 seconds. Our range analysis
%performs better than sampling in 23 out of 25 benchmarks for the maximal
%distance, and for 1 benchmark the distance is the same. In average, \Tool
%perform better than sampling for 15 benchmarks.
%
%For the error distributions, \Tool error analysis performs better than sampling
%for 11 benchmarks for the maximal absolute distance. On average, sampling is
%always closer to the golden distribution than \Tool.
%
%We observe a better accuracy for \Tool for the range analysis with respect to
%the error analysis. The reason for this slight imprecision is the high
%dependency between operands in the computation of the absolute error
%distribution.
%
%However, we note how close both \Tool and sampling are to the golden model.

