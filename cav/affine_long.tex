\section{Symbolic Affine Arithmetic}
\label{sec:symbolicaffine}

In this section, we introduce \emph{symbolic affine arithmetic}, which we
employ to generate the symbolic form for the roundoff error that we use in
\cref{subsec:conderror}.
%
Affine arithmetic~\cite{affineoriginal} is a model for range analysis that extends classic interval arithmetic~\cite{intervalarithmetic} with information about linear correlations between operands.
%
Symbolic affine arithmetic extends standard affine arithmetic by keeping the coefficients of the noise terms \emph{symbolic}.
%
%Thus, we use a single quote to distinguish a symbolic operation $(f(x))^{\prime}$ from its concrete counterpart $f(x)$. In other words,  all the operands and operators in $(f(x))^{\prime}$ are symbolic. 
%
We define a \emph{symbolic affine form} as
%
\begin{align}
\hat{x}= x_{0}+\sum_{i=1}^{n} x_{i}\epsilon_{i},\qquad\text{where}\; \epsilon_{i} \in [-1, 1]. \label{eq:affineform}
\end{align}
%
We call $x_{0}$ the central symbol of the affine form, while $x_{i}$ are the symbolic coefficients for the noise terms $\epsilon_{i}$.
We can always convert a symbolic affine form to its corresponding interval representation. 
%
This can be done using interval arithmetic or, to avoid precision loss, using a
global optimizer.
%

Affine operations between symbolic forms follow the usual rules, such as
\begin{align*} 
\alpha\hat{x}+\beta\hat{y}+\zeta=\alpha{x_{0}}+\beta{y_{0}}+\zeta+\sum_{i=1}^{n}{(\alpha x_{i}+\beta y_{i})\epsilon_{i}} 
\end{align*}
%
Non-linear operations cannot be represented exactly using an affine form.
%
Hence, we approximate them like in standard affine arithmetic~\cite{affinebook}.
%

\noindent \textbf{Sound Error Analysis with Symbolic Affine Arithmetic.}
We now show how the roundoff errors get propagated through the four basic arithmetic operation.
%
We apply these propagation rules to an arithmetic expression
to accurately keep track of the roundoff errors. 
%
Since the (absolute) roundoff error directly depends on the range of a computation,  we describe range and error together as a pair
%
\todo[inline]{Fred: terminology question: this last statement is only true of the \emph{absolute} error.  Is this what \emph{roundoff error} means? If yes, is there another conventional name for the \emph{realtive} error.}
\texttt{(range: Symbol, $\widehat{err}$: Symbolic Affine Form)}.
Here, \texttt{range} represents the infinite-precision range of the computation, while $\widehat{err}$ is the symbolic affine form for the roundoff error in floating-point precision.
Unary operators (e.g., rounding) take as input a (range, error form) pair, and return a new output pair;
binary operators take as input two pairs, one per operand.
%
For linear operators, the ranges and errors get propagated using the standard rules of affine arithmetic.
%

For the multiplication, we distribute each term in the first operand to every term in the second operand:
\begin{align}
\nonumber
(\texttt{x},\:\widehat{err}_x)*(\texttt{y}, \:\widehat{err}_y)=(\texttt{x*y},\:\texttt{x}*\widehat{err}_y + \texttt{y}*\widehat{err}_x + \widehat{err}_x*\widehat{err}_y)
\end{align}
The output range is the product of the input ranges and the remaining terms contribute to the error.
% The remaining three terms contribute to the output error.
%
Only the last (quadratic) expression cannot be represented exactly in symbolic affine arithmetic;
we bound such non-linearities using a global optimizer.
%
The division is computed as the term-wise multiplication of the numerator with the inverse of the denominator.
%
Hence, we need the inverse of the denominator error form, and then we can proceed as for multiplication.
%
To compute the inverse, we leverage the symbolic expansion used in FPTaylor~\cite{solovyev2018rigorous}.

Finally, after every operation we apply the unary rounding operator from \cref{eq:traditional}. The infinite-precision range is not affected by rounding. The rounding operator appends a fresh noise term to the symbolic error form. 
The coefficient for the new noise term is the (symbolic) floating-point range given by the sum of the input range with the input error form.